\documentclass[12pt,a4paper]{article}

% Fonts
\usepackage{libertine}
\usepackage[libertine]{newtxmath}
\usepackage[scaled=.8]{Fira Mono}

% KableExtra support
\usepackage{booktabs}
\usepackage{longtable}
\usepackage{array}
\usepackage{multirow}
\usepackage{wrapfig}
\usepackage{colortbl}
\usepackage{pdflscape}
\usepackage{tabu}
\usepackage{threeparttable}
\usepackage{threeparttablex}
\usepackage[normalem]{ulem}
\usepackage{makecell}
\usepackage{etoolbox}
\usepackage{tocloft}

% Colours
\usepackage[usenames,dvipsnames]{xcolor}
\definecolor{darkblue}{rgb}{0.0,0.0,0.55}

% Spacing
\usepackage{setspace}

% Margins
\usepackage[margin=2cm]{geometry}

% Packages I've been using for different reasons
\usepackage{hyperref}
\usepackage{dcolumn}
\usepackage{graphicx}
\usepackage{float}
\floatplacement{figure}{H}
\usepackage{pgf}
\usepackage{tikz}
\usetikzlibrary{arrows}
\usetikzlibrary{positioning}
\usepackage{mathtools}
\usepackage{caption}

% English
\usepackage[english]{babel}
\usepackage[english]{isodate}
\cleanlookdateon

% Penalties
\exhyphenpenalty=1000
\hyphenpenalty=1000
\widowpenalty=1000
\clubpenalty=1000

% Hypersetup
\hypersetup{
  linkcolor=Mahogany,
  citecolor=Mahogany,
  urlcolor=darkblue, 
  breaklinks=true, 
  colorlinks=true,
      pdfauthor={},
      pdfkeywords={Brazil, crime, extralegal violence, lynching, vigilantism},
  }

% Bibliography
\usepackage{natbib}
\bibliographystyle{apalike}
\makeatletter
% Remove comma after author
\setcitestyle{aysep={}}
\patchcmd{\NAT@citex}
	  {\@citea\NAT@hyper@{%
		 \NAT@nmfmt{\NAT@nm}%
		 \hyper@natlinkbreak{\NAT@aysep\NAT@spacechar}{\@citeb\@extra@b@citeb}%
		 \NAT@date}}
	  {\@citea\NAT@nmfmt{\NAT@nm}%
	   \NAT@aysep\NAT@spacechar\NAT@hyper@{\NAT@date}}{}{}
	\patchcmd{\NAT@citex}
	  {\@citea\NAT@hyper@{%
		 \NAT@nmfmt{\NAT@nm}%
		 \hyper@natlinkbreak{\NAT@spacechar\NAT@@open\if*#1*\else#1\NAT@spacechar\fi}%
		   {\@citeb\@extra@b@citeb}%
		 \NAT@date}}
	  {\@citea\NAT@nmfmt{\NAT@nm}%
	   \NAT@spacechar\NAT@@open\if*#1*\else#1\NAT@spacechar\fi\NAT@hyper@{\NAT@date}}
	  {}{}
% Patch case where name and year are separated by aysep
\patchcmd{\NAT@citex}
  {\@citea\NAT@hyper@{%
     \NAT@nmfmt{\NAT@nm}%
     \hyper@natlinkbreak{\NAT@aysep\NAT@spacechar}{\@citeb\@extra@b@citeb}%
     \NAT@date}}
  {\@citea\NAT@nmfmt{\NAT@nm}%
   \NAT@aysep\NAT@spacechar\NAT@hyper@{\NAT@date}}{}{}
% Patch case where name and year are separated by opening bracket
\patchcmd{\NAT@citex}
  {\@citea\NAT@hyper@{%
     \NAT@nmfmt{\NAT@nm}%
     \hyper@natlinkbreak{\NAT@spacechar\NAT@@open\if*#1*\else#1\NAT@spacechar\fi}%
       {\@citeb\@extra@b@citeb}%
     \NAT@date}}
  {\@citea\NAT@nmfmt{\NAT@nm}%
   \NAT@spacechar\NAT@@open\if*#1*\else#1\NAT@spacechar\fi\NAT@hyper@{\NAT@date}}
  {}{}
\makeatother

% Make links footnotes instead of hotlinks:
%  \setlength{\emergencystretch}{3em}  % prevent overfull lines
 \providecommand{\tightlist}{%
   \setlength{\itemsep}{0pt}\setlength{\parskip}{0pt}}
   
% Numbered sections
\setcounter{secnumdepth}{5}
% % % Redefines (sub)paragraphs to behave more like sections
% \ifx\paragraph\undefined\else
% \let\oldparagraph\paragraph
% \renewcommand{\paragraph}[1]{\oldparagraph{#1}\mbox{}}
% \fi
% \ifx\subparagraph\undefined\else
% \let\oldsubparagraph\subparagraph
% \renewcommand{\subparagraph}[1]{\oldsubparagraph{#1}\mbox{}}
% \fi
 
% Spacing
\doublespacing

% Title
\title{Vigilantism and Institutions:\\ Understanding Attitudes Toward Lynching in Brazil\footnote{We thank Rosario Aguilar, Louise Araújo, Nicholas Barnes, Hannah Baron, Dara Kay Cohen, Nicholas Cowen, Daniel J. D'Amico, Miriam Golden, Richard Locke, Umberto Mignozzetti, Eduardo Moncada, Irfan Nooruddin, Jonathan Obert, Brian J. Phillips, Catarina Roman, Alexandra Scacco, Livia Schubiger, Natán Skigin, Nicholas Rush Smith, Richard Snyder, and Georg Vanberg for their valuable comments. We received helpful feedback in seminars at EGAP, the Extra-legal Governance Institute at the University of Oxford, the Comparative Workshop at Brown University, the Kroc-Kellogg Peace, Conflict, Crime and Violence Workshop at Notre Dame, and the Severyns Ravenholt Seminar in Comparative Politics at the University of Washington. This study received IRB approval from Brown University in October 2020 (Protocol 2009002803). Replication materials are available at \url{https://github.com/danilofreire/lynching-experiment-brazil}. We thank the Centre for the Study of Governance \& Society at King's College London and the Templeton Foundation for financial support.}}

% Author
\author{Danilo Freire\footnote{School of Social and Political Sciences, University of Lincoln, \href{mailto:danilofreire@gmail.com}{\texttt{danilofreire@gmail.com}}, \url{https://danilofreire.github.io}.} \and David Skarbek\footnote{Department of Political Science, Brown University, \href{mailto:david_skarbek@brown.edu}{\texttt{david\_skarbek@brown.edu}}, \url{http://davidskarbek.com}.}}

% Date
\date{\today}

% Begin document
\begin{document}
\maketitle

% Abstract
\begin{abstract}
\doublespacing \noindent Why do people support extrajudicial violence? In two survey experiments with respondents in Brazil, we examine which characteristics of lynching scenarios garner greater support for lynching and whether providing different types of information about lynching reduces support for it. We find that people often do support community members to take vengeance. In particular, our analysis finds that people strongly support the use of extrajudicial violence by families of victims against men who sexually assault and murder women and children. We also find that criminal punishment and the threat of vendettas reduce support, but appeals to the human rights of victims have zero effect on support for lynchings. Unlike the U.S. experience with lynchings, race was not observed to play an important role in how respondents answered the survey.
\vspace{.25cm}

\noindent \textbf{Keywords:} Brazil, crime, extralegal violence, lynching, vigilantism
\vspace{.25cm}

\end{abstract}

\newpage

\section*{Introduction}
\label{sec:introduction}

\doublespacing

The regulation of violence is a crucial function of the state. When states
control violence effectively, markets, politics, and civil society can flourish
\citep{besley2011pillars, north2009violence}. However, states still struggle to
maintain the monopoly of force in much of the Global South. Scholars have
written extensively about how rebel organizations build support to challenge
government institutions, yet we know little about what motivates local
extrajudicial violence and vigilantism in the developing world
\citep{bateson2020politics}.\footnote{On vigilantism more generally, see
\citet{cohen2022collective}, \citet{schuberth2013challenging},
\citet{smith2019contradictions}, and \citet{zizumbo2017community}.} This
omission is significant not only because local violence is pervasive in weak
states, but also because it possibly comes at a greater social cost than
terrorism or wars \citep{blair2017predicting}. 

One of the most serious forms of extrajudicial violence is lynching. Lynching
can be defined as ``incidents of physical violence committed by large numbers
of private citizens against one or more individuals accused of having committed
a ``criminal'' offense, whether or not this violence resulted in the death of
the victim(s)'' \citep[645]{godoy2004justice}. Although lynchings occur in more
than one hundred countries in all regions of the world
\citep{jung2020lynching,smith2019contradictions}, Latin America has been
particularly affected by a sharp increase in vigilante violence. Lynching
episodes have been reported in Guatemala, Mexico, Peru, Venezuela and other
countries in the region \citep{barbara2015vigilantes, cruz2019determinants,
godoy2004justice}. From 2011 to 2015, Brazil registered about 2,500 lynching
episodes, and 173 people were killed by angry mobs in 2015 alone---nearly one
execution every two days \citep{barbara2015vigilantes, oliveira2016mob}.
According to José de Souza Martins \citeyearpar{martins2015linchamentos}, who
has studied lynchings in Brazil for more than thirty years, these figures are
not only the highest in the country's history, but among the highest in the
world. The people who participate in lynchings are typically young men, but
they also sometimes include teenage women and girls, elderly women, and even
members of the local police \citep{moura2017linchamentos}.

These alarming figures suggest that, despite being dismissed by policymakers as
irrational and senseless, vigilantism enjoys considerable approval in Latin
America \citep{berg2011globalizing, goldstein2005flexible}. This raises a
number of puzzles: under which conditions do citizens perceive lynchings as a
legitimate response to crime? Are some groups seen as more ``deserving'' of
lynchings that others? And how can we reduce pro-vigilantism attitudes in
young, unequal democracies?

In this study, we run two survey experiments and use primary qualitative data
to understand attitudes toward lynchings in Brazil. Our paper begins with a
conjoint experiment that evaluates in which circumstances citizens perceive
lynchings as a justifiable punishment for crime. We also included open-ended
questions for respondents to elaborate on their decision process. Lastly, we
employ an information provision experiment to test the efficacy of three
strategies to reduce individual support for lynchings.

Our empirical approach offers several distinct advantages. First, the data we
collected come from the same level of analysis as the phenomenon itself: local
residents. Second, online surveys are less susceptible to social desirability
bias, and survey experiments help to elicit sensitive information---an
especially important benefit in researching this topic
\citep{horiuchi2022does}. Third, the experimental nature of the approach allows
us to precisely estimate effect sizes and compare different theoretical
explanations simultaneously. Finally, we can also control for confounding,
which is a potential concern in previous research on vigilantism.

Our results show that people's attitudes toward lynching are shaped strongly by
the characteristics of the incident that gives rise to it. In particular,
respondents express support for lynchings perpetrated by the family of a victim
of a sexual offense or murder. Subgroup analyses show that our findings are
consistent across different segments of the population. In line with
\citet{godoy2006popular} and \citet{martins2015linchamentos}, we find no
experimental or qualitative evidence that lynching support is primarily driven
by the race of the criminal. Our results also indicate that popular support for
lynchings can be substantially reduced by reminding individuals of the
penalties for lynching and of the risk of sparking vendettas, but not by
emphasizing human rights protections. This suggests that campaigns focusing on
the risks to vigilantes' personal security can be effective correctives to
pro-lynching attitudes and may strengthen support for the rule of law. 

%%%%%%%%%%%%%

\section*{When Is Lynching Perceived as More Justified?}
\label{sec:exp01}

We invited respondents from all regions of Brazil to participate in our
experiment. Qualtrics recruited 2,406 adult Brazilian citizens between October
30 and December 22, 2020 to take part in our survey experiment.\footnote{Our
ethics statement is available in Section E of the Supplementary Material.} We
included ten questions on demographics and other information that might
influence the results. These include the respondent's age, gender, ethnicity,
level of education, monthly family income, political orientation (left to
right), support for the death penalty, whether they had been victimized in the
last year, their trust in the police, and their trust in the judicial system.
We used these measures to disentangle heterogeneous effects in the main
responses. Descriptive statistics are available in Section B of the
Supplementary Material.

In our first experiment, we used a choice-based conjoint experimental design.
We presented respondents with five pairs of profiles. Each profile consisted of
eight attributes: (1) gender of the crime perpetrator; (2) age of the crime
perpetrator; (3) race of the crime perpetrator; (4) residency of the crime
perpetrator (local or nonlocal); (5) type of offense; (6) gender of the victim
of the crime; (7) age of the crime victim; (8) type of lynching perpetrator.
The attributes and levels are displayed in table \ref{tab:categories}.

In addition to our knowledge of the Brazilian case, we also chose our
attributes based partly on existing work on crime, vigilantism, and
extrajudicial violence. From the prison violence literature, we know that the
relative age differences and the genders of perpetrators and victims affect how
it is perceived \citep{fleisher2009myth}. We likewise know that certain
offenses are seen as especially reprehensible \citep{skarbek2014social}, so we
included such offenses, like molestation, alongside less serious offenses. We
include race because, in the United States, violent lynchings have often been
wielded for racist reasons \citep{dray2003hands}. Likewise, we include
residency because extrajudicial violence is often used against people ``who
don't belong'' or are ``outsiders'' in some respect. Finally, in honor cultures
that value retaliatory violence, people believe that victims and the family of
victims have a special right (and often an obligation) to enact retribution
\citep{weiner2013rule}.


\vspace{.3cm}

\begin{table}[htpb]
\begin{center}
\caption{Attributes and levels}
\label{tab:categories}
\begin{tabular}{l !{\vrule width 1pt}p{9cm}}
\Xhline{2\arrayrulewidth}
\textbf{Attribute} & \multicolumn{1}{c}{\textbf{Levels}} \\
\Xhline{2\arrayrulewidth}
Gender of crime perpetrator & Male; female \\ [4pt]
Age of crime perpetrator & Teenager; adult; elderly \\ [4pt]
Race of crime perpetrator & Black; white; Native Brazilian; Asian \\ [4pt]
Residency of crime perpetrator & Resident in the community; outsider \\ [4pt]
Offense & Picks pockets; steals cars; molests; rapes; murders \\ [4pt]
Gender of crime victim & Male; female\\ [4pt]
Age of crime victim & Child; teenager; adult; elderly\\ [4pt]
Lynching perpetrators & Bystanders; neighbors; family of the victim; gangs; police \\
\Xhline{2\arrayrulewidth}
\end{tabular}
\end{center}
\end{table}

Respondents were asked to indicate in which scenario in each pair they thought
that extrajudicial punishment was more justified. We also included a text box
below each experiment for subjects to explain why they selected a particular
case and whether they believed lynching was justified in neither case.
Respondents read the following prompt before they started the
experiment:\footnote{The vignettes were presented in Portuguese. The original
text is available in Section C.1 of the Supplementary Material.}

\begin{itemize} \tightlist \item Lynchings are often used as social punishment
  in Brazil. Lynchings are cases in which three or more people physically
  attack or execute a suspected criminal in public. We are interested in
  knowing more about how Brazilians see these episodes. In the next five
  questions, please read the description of two possible lynching victims in
  Brazil and indicate in which case you believe the punishment is more
  justified. Even if you are not entirely sure, please select one of the cases.
\end{itemize}

We report marginal means in our main analysis instead of average marginal
component effects (AMCEs). \citet{leeper2020measuring} show that AMCEs can be
misleading in subgroup comparisons as model results are sensitive to the choice
of reference categories in interactions. In contrast, marginal means allow for
a clear description of quantities of interest, in our case preferences about
lynching, while also allowing for easy comparisons between groups of
respondents. In a forced-choice experiment such as this one, a marginal-means
estimate of 0.5 indicates that respondents are indifferent to this attribute
vis-à-vis other attributes. When the coefficient is lower than 0.5, respondents
dislike profiles with this attribute. Conversely, when the point estimate is
higher than 0.5, respondents prefer profiles containing a given attribute.

Figure \ref{fig:exp01} shows the main results for the conjoint experiment. The
graph illustrates the preference associated with each attribute of the
hypothetical lynching episode. Dots with horizontal bars represent point
estimates and 95 percent confidence intervals from linear regressions with
robust standard errors clustered at the respondent level.\footnote{Based on a
semisupervized Dirichlet allocation model, we estimated about 20 percent of
respondents were opposed to lynchings in general but forced to choose. Please
refer to Section C.5 of the Supplementary Material for further information.}

\begin{figure}[ht]
\includegraphics[width=17cm]{conjoint-color.pdf}
\caption{Relative preferences for lynching-episode attributes (marginal means)}
\centering
\label{fig:exp01}
\end{figure}

Respondents selected male offenders as preferred lynching victims more often
than female offenders. We found identical results in our subgroup analyses, in
which we disaggregated our sample according to respondents' gender, income,
race, support for the death penalty, and views on the police and judiciary. In
all cases, men and teenage boys were more likely to be selected as more
justified victims of lynching. Compared with the attributes we discuss below,
the gender of the crime perpetrator had one of the largest effects in our
experiment. With regard to the age of the crime perpetrator, respondents
tolerated the lynching of adults more than attacks against teenagers or the
elderly. More precisely, subjects were indifferent toward teenagers but reacted
positively to lynching of adults and negatively to lynching of elderly people.

When we analyzed the effect of the race of the crime perpetrator, in contrast
with the American experience, respondents favored the lynching of white
perpetrators the most. Of all the racial identities that we included in the
experiment, Black people were the least likely to be chosen as a more justified
lynching victim, although the 95 percent confidence intervals overlap with the
remaining categories. The effect is also robust for most subgroups, including
white respondents. Asian respondents, however, rated fellow Asians as their
least preferred choice. Taken together, the results provide experimental
evidence that support for lynchings in Brazil does not resemble the typical
racial patterns scholars have observed in the United States
\citep{dray2003hands, obert2018keeping, seguin2019national}. These findings are
consistent with recent research on vigilantism in Haiti
\citep{jung2020lynching} and with journalistic observations. They further
suggest that the US experience with lynchings might be distinct from
vigilantism in other places and times \citep{oliveira2016mob}. 

The residency of the perpetrator is not a significant driver of support for
lynching. Subjects were indifferent as to whether the crime perpetrator lives
in the area or in another neighborhood. In this sense, it is possible that
subjects see lynchings not as a means to protect the area in which they live,
which people commonly suppose vigilante groups do, but rather as retribution in
specific cases. 

The next attribute is type of offense. Rape was by far the most significant
attribute in our model, followed by murder and molestation. It is not
surprising that murder and rape were the most important offenses to
respondents. Overall, we found that respondents do not believe that property
crimes, such as car theft and pickpocketing, provide justification for
lynching. When we analyzed the data by groups, we also saw that poorer
respondents, whose household income is less than R\$3,000 (US\$650) per month,
are more likely to support the lynching of individuals who commit a property
crime. 

For age, we observed that respondents were particularly concerned about
children as crime victims, as this attribute is strongly correlated with
support for lynchings. Respondents also thought lynchings are more justified
when the victim is a teenager, an elderly person, or an adult.

When we analyzed the gender of the crime victim, we found that respondents
support lynchings more strongly when the victim is a woman or teenage girl and
less strongly when the lynching victim is a man or teenage boy. The difference
here is smaller than when we considered the gender of the crime perpetrator,
but as in the previous estimation the 95 percent confidence intervals of each
attribute did not overlap. 

Lastly, and related to this, we examined who respondents believed are more
legitimate perpetrators of lynchings. Respondents considered the family to be
the most justified in carrying out a lynching. Again, the intervals here did
not overlap with those of any other category. Bystanders, gangs, and neighbors
do not increase or decrease respondents attitudes about lynching. If the police
carry out a lynching, that reduced the perceived justification. In sum,
respondents did not believe that lynchings should be carried out by the state
but did believe they should be used as a tool for individual or family
retribution. These results are consistent with norms of an honor culture in
which offenses are seen to tarnish the victim's status and the only way to
remove the stigma is through self-help efforts in the form of retaliatory
violence \citep{nisbett2018culture}.

Respondents were given the opportunity to explain the reason why they answered
as they did. We analyzed a random sample of nearly 17 percent of the answers.
Three major themes emerge. First, respondents felt that crimes that were
especially heinous deserved a response. For example, one respondent explained
their choice of a particular case as more justified by noting, ``Eye for an
eye, it's absolutely fair.'' A second  theme is the injustice of victimizing
vulnerable populations. One respondent explains, ``The victim is an elderly
person, it's okay to lynch the criminal.'' Another respondent says, ``If the
victim is a child, lynchings are not enough, they deserve more.'' Finally, a
third theme is that families have a special right to vengeance. One respondent
explained their choice by saying, ``The family was appalled and did justice
with their own hands!'' These open-ended answers are consistent with the
quantitative results about what people believe, but they also provide
additional evidence about why they hold their beliefs.\footnote{Please refer to
Section C.5 of the Supplementary Materials for further information.} 

%%%%%%%%%%%%%%%%%%%%%
\section*{How to Reduce Support for Lynchings}
\label{sec:exp03}

Even if people view lynchings as a legitimate activity, that does not mean that
this norm cannot be changed. In our second experiment, we measured the effect
of information provision on attitudes about lynching. More specifically, we
tested whether reminding respondents about the legal and social consequences of
vigilante justice reduces subjects' support for it.\footnote{To prevent
carryover effects caused by previous exposure to the conjoint, we randomized
the order of experiments in our survey.} The experiment included three
treatment conditions and a control group. In all of them we presented
respondents with a short statement affirming that some Brazilians support
vigilantism under certain conditions. We asked the control group to rate their
agreement with the statement. Respondents were asked to use 0 to 49 if they
disagreed, 50 if they neither agreed nor disagreed, and 50--100 if they agreed
with the statement.

Each of the three treatment groups received a different message about the legal
or social consequences of lynching in Brazil. In the first treatment, we
informed subjects about how the Brazilian constitution and penal code punishes
civilian violence. The second treatment group was notified about the human
rights guarantees enshrined in Brazil's legal framework. The last treatment
noted that lynchings can spark vendettas and initiate a cycle of violence in
the community. Subjects in the control group received no information about the
consequences of lynchings. The text shown to the control and treatment groups
can be read below.\footnote{The Portuguese version is available in Section D.1
of the Supplementary Material.} 

\begin{itemize}
\tightlist
\item
  \emph{Control group:} In Brazil, some people believe that lynching may be justified under certain conditions. To what degree do you agree or disagree that lynching can be justified? Please use the slider below to indicate your preference. For disagreement, use 0 to 49; for agreement, use 51 to 100. Please use 50 if you neither agree nor disagree.
\end{itemize}

\begin{itemize}
\tightlist
\item
  \emph{Treatment 1 (Legal punishment for lynching perpetrators):} In Brazil, some people believe that lynching may be justified under certain conditions. \textbf{However, the Brazilian constitution and penal code strictly forbid lynching and those involved can be accused of torture or murder}. To what degree do you agree or disagree that lynching can be justified? Please use the slider below to indicate your preference. For disagreement, use 0 to 49; for agreement, use 51 to 100. Please use 50 if you neither agree nor disagree.
\end{itemize}

\begin{itemize}
\tightlist
\item
  \emph{Treatment 2 (Human rights):} In Brazil, some people believe that lynching may be justified under certain conditions. \textbf{However, the Brazilian constitution states that all individuals have the right of not being tortured, including criminals}. To what degree do you agree or disagree that lynching can be justified? Please use the slider below to indicate your preference. For disagreement, use 0 to 49; for agreement, use 51 to 100. Please use 50 if you neither agree nor disagree.
\end{itemize}

\begin{itemize}
\tightlist
\item
  \emph{Treatment 3 (Vendettas):} In Brazil, some people believe that lynching may be justified under certain conditions. \textbf{However, lynchings can trigger a new cycle of violence as the family or friends of the victim may retaliate against the community}. To what degree do you agree or disagree that lynching can be justified? Please use the slider below to indicate your preference. For disagreement, use 0 to 49; for agreement, use 51 to 100. Please use 50 if you neither agree nor disagree.
\end{itemize}

We are interested in the difference between the average score given by each of
the treatment groups and the average score given by the control group. We
estimate average treatment effects using OLS with dummy indicators for the
treatment groups.

\vspace{.5cm}

\begin{table}[ht] \centering
  \caption{Average treatment effects for experiment 2}
  \label{tab:exp02}
\begin{tabular}{@{\extracolsep{3pt}}lD{.}{.}{-3} D{.}{.}{-3} D{.}{.}{-3} D{.}{.}{-3} }
\\[-1.8ex]\hline \\[-1.8ex]
\\[-1.8ex] & \multicolumn{4}{c}{\textbf{Lynching Support}\vspace{.5cm}} \\
\\[-1.8ex] & \multicolumn{1}{c}{(1)} & \multicolumn{1}{c}{(2)} & \multicolumn{1}{c}{(3)} & \multicolumn{1}{c}{(4)}\\
\hline \\[-1.8ex]
 Constitution and penal code & -4.509^{**} &  &  &  \\
  & (1.805) &  &  &  \\
  Human rights &  & -1.571 &  &  \\
  &  & (1.801) &  &  \\
  Vendettas &  &  & -3.156^{*} &  \\
  &  &  & (1.879) &  \\
  Combined treatments &  &  &  & -3.023^{**} \\
  &  &  &  & (1.493) \\
  Constant & 40.823^{***} & 40.823^{***} & 40.823^{***} & 40.823^{***} \\
  & (1.293) & (1.293) & (1.293) & (1.293) \\
 N & \multicolumn{1}{c}{1,114} & \multicolumn{1}{c}{1,173} & \multicolumn{1}{c}{1,092} & \multicolumn{1}{c}{2,215} \\
\hline \\[-1.8ex]
\multicolumn{5}{l}{$^{*}$p $<$ .1; $^{**}$p $<$ .05; $^{***}$p $<$ .01} \\
\multicolumn{5}{l}{Robust standard errors in parentheses.} \\
\end{tabular}
\end{table}
\normalsize

Table \ref{tab:exp02} summarizes our main results for this experiment. We find
that when respondents are reminded of legal punishments associated with
lynchings, their support for lynching decreases by about 11 percent. This
result is consistent with the idea that respondents will support an activity
less if they perceive it to be more costly. The next treatment group was told
the human rights message. The result is not statistically significant and
indicates that appeals to individual liberties and the rule of law are likely
to fall short. This finding has practical implications, as many campaigns
against violent crimes are conducted by human rights groups, which tend to
emphasize this type of message. However, participants do not generally believe
that offenders do, in fact, have human rights. Accordingly, we have little
reason to believe that such messaging is effective. By contrast, Treatment 1
reminds respondents that perpetrators of lynching potentially face a high
personal cost from doing so.
 
Our last treatment, which raises the possibility of sparking vendettas, has a
large, negative effect on support for lynchings. Mentioning that lynchings
trigger cycles of violence has a negative effect of about 8 percent when
compared with the control group. While we expected that the coefficient for
this treatment condition would be larger than that for the treatment on legal
punishments, the means of both groups are not statistically different from each
other. As with the other experiments, the results remain stable when we conduct
subgroup analyses. 

\section*{Conclusion}
\label{conclusion}

Regulating violence is one of the most fundamental responsibilities of the
state, and in many places in the world today, states are failing to do so.
Brazil provides a telling example of how extrajudicial violence can escalate
and pose a serious threat to the stability of government institutions. 

Our results show that the characteristics of a lynching significantly affect
people's support for it, especially when the crimes are heinous or committed
against vulnerable populations. In contrast with studies about mob violence in
the American South, we find no evidence that lynchings are strictly racially
motivated in Brazil. While we were able to reduce support for lynchings by
reminding respondents that lynchings are a criminal act and that lynchings risk
igniting vendettas, appeals to human rights had no effect in our experiment. 

These findings have implications that extend well beyond Brazil. In particular,
our experiments indicate that lynchings have a crucial gendered aspect, which
has also been documented in other contexts. Using data from 18 Latin American
countries, \citet{nivette2016institutional} also finds that respondents are
most likely to support lynchings when the criminal raped a child, and studies
about lynchings in the American South point out that several episodes resulted
from accusations of sexual assault \citep{jacquet2013giles, smaangs2020race}.
We believe that a culture of honor may explain these results. Individuals in
honor societies view crime as an attack on their personal reputation and, in
turn, are more likely to take revenge to defend their status and that of people
perceived as deserving protection, such as women and children
\citep{nisbett2018culture}. A culture of honor may also explain why Brazilians
see lynchings carried out by the family of the victim as more justified, as
well as refraining from using extralegal violence if it can trigger vendettas.
We expect similar results in societies which share those cultural norms.
Lastly, the fact that race does not appear to be a major motivation behind
lynchings also reflects the experience of places like Haiti or Southern Africa,
where popular violence was mainly driven by other social factors
\citep{berg2011globalizing, jung2020lynching}. Even in the American South,
Whites and Blacks also lynched people of their own race \citep{beck1997race}.
In this respect, our paper also highlights that lynchings in the Global South
may be more strongly connected with the idea of ``popular justice'' than with
racial animus \citep{martins2015linchamentos}.

Our work also suggests new avenues for further research. While previous studies
indicate that there is a relationship between pro-lynching attitudes and actual
engagement in vigilante violence \citep{weisburd1988vigilantism}, future
research should try to identify which factors motivate citizens to participate
in lynchings episodes \citep{nivette2016institutional}. Moreover, it is still
unclear whether other forms of vigilantism, such as security patrols, property
damages, or coercive interrogations have similar levels of popular support
\citep{bateson2020politics}. Finally, lynchings have often been described as a
response to low state capacity \citep{trevizo2022mexico}, yet in some cases
state agents actively incite or engage in vigilante violence themselves
\citep{arias2010violent}. Future studies may investigate in which circumstances
state authorities favor informal local control instead of public law
enforcement. Unravelling these issues is crucial if scholars wish to better
understand how and why states and civil society decide to use violence.

\setlength{\parindent}{0cm}
\setlength{\parskip}{5pt}

% Bibliography
\bibliography{references.bib}

\end{document}

