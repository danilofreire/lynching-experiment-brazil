% !TEX program = XeLaTeX
\documentclass[12pt,a4paper,]{article}

% Fonts
\usepackage{libertine}
% This is a nice mathfont, it fits well with Libertine
% Comment the line if you want to use a different one
\usepackage[libertine]{newtxmath}
% The default monofont 
% Again, comment the line below if you want to change it
\usepackage[scaled=.95]{inconsolata}

% Add the following packages to support kableExtra
\usepackage{booktabs}
\usepackage{longtable}
\usepackage{array}
\usepackage{multirow}
\usepackage{wrapfig}
\usepackage{colortbl}
\usepackage{pdflscape}
\usepackage{tabu}
\usepackage{threeparttable}
\usepackage{threeparttablex}
\usepackage[normalem]{ulem}
\usepackage{makecell}
\usepackage{etoolbox}
\usepackage{tocloft}

% Dots in toc
\renewcommand{\cftsecleader}{\cftdotfill{\cftdotsep}}

% Colours
\usepackage[usenames,dvipsnames]{xcolor}
\definecolor{darkblue}{rgb}{0.0,0.0,0.55}

% Spacing
\usepackage{setspace}

% Margin
\usepackage[margin=2cm]{geometry}

% Packages I've been using for different reasons...
\usepackage{hyperref}
\usepackage{dcolumn}
\usepackage{graphicx}
\usepackage{float}
\floatplacement{figure}{H}
\usepackage{pgf}
\usepackage{tikz}
\usetikzlibrary{arrows}
\usetikzlibrary{positioning}
\usepackage{mathtools}
\usepackage{caption}

% UK English
\usepackage[UKenglish]{babel}
\usepackage[UKenglish]{isodate}
\cleanlookdateon

% Penalties
\exhyphenpenalty=1000
\hyphenpenalty=1000
\widowpenalty=1000
\clubpenalty=1000

% Hypersetup
\hypersetup{
  linkcolor=Mahogany,
  citecolor=Mahogany,
  urlcolor=darkblue, 
  breaklinks=true, 
  colorlinks=true,
      pdfauthor={Danilo Freire; David Skarbek},
      pdfkeywords={Brazil, crime, extralegal violence, lynching, vigilantism},
  }

% If XeTex, LuaLaTeX, etc
\usepackage{ifxetex,ifluatex}
\usepackage{fixltx2e} % provides \textsubscript
\ifnum 0\ifxetex 1\fi\ifluatex 1\fi=0 % if pdftex
  \usepackage[T1]{fontenc}
  \usepackage[utf8]{inputenc}
\else % if luatex or xelatex
  \ifxetex
    \usepackage{amssymb,amsmath}
    \usepackage{mathspec}
  \else
    \usepackage{fontspec}
  \fi
  \defaultfontfeatures{Ligatures=TeX,Scale=MatchLowercase}
\fi
% use upquote if available, for straight quotes in verbatim environments
\IfFileExists{upquote.sty}{\usepackage{upquote}}{}
% use microtype if available
\IfFileExists{microtype.sty}{%
\usepackage{microtype}
\UseMicrotypeSet[protrusion]{basicmath} % disable protrusion for tt fonts
}{}

% Language

% Bibliography
\usepackage{natbib}
\bibliographystyle{apalike}
\makeatletter
% Remove comma after author
\setcitestyle{aysep={}}
\patchcmd{\NAT@citex}
	  {\@citea\NAT@hyper@{%
		 \NAT@nmfmt{\NAT@nm}%
		 \hyper@natlinkbreak{\NAT@aysep\NAT@spacechar}{\@citeb\@extra@b@citeb}%
		 \NAT@date}}
	  {\@citea\NAT@nmfmt{\NAT@nm}%
	   \NAT@aysep\NAT@spacechar\NAT@hyper@{\NAT@date}}{}{}
	\patchcmd{\NAT@citex}
	  {\@citea\NAT@hyper@{%
		 \NAT@nmfmt{\NAT@nm}%
		 \hyper@natlinkbreak{\NAT@spacechar\NAT@@open\if*#1*\else#1\NAT@spacechar\fi}%
		   {\@citeb\@extra@b@citeb}%
		 \NAT@date}}
	  {\@citea\NAT@nmfmt{\NAT@nm}%
	   \NAT@spacechar\NAT@@open\if*#1*\else#1\NAT@spacechar\fi\NAT@hyper@{\NAT@date}}
	  {}{}
% Patch case where name and year are separated by aysep
\patchcmd{\NAT@citex}
  {\@citea\NAT@hyper@{%
     \NAT@nmfmt{\NAT@nm}%
     \hyper@natlinkbreak{\NAT@aysep\NAT@spacechar}{\@citeb\@extra@b@citeb}%
     \NAT@date}}
  {\@citea\NAT@nmfmt{\NAT@nm}%
   \NAT@aysep\NAT@spacechar\NAT@hyper@{\NAT@date}}{}{}
% Patch case where name and year are separated by opening bracket
\patchcmd{\NAT@citex}
  {\@citea\NAT@hyper@{%
     \NAT@nmfmt{\NAT@nm}%
     \hyper@natlinkbreak{\NAT@spacechar\NAT@@open\if*#1*\else#1\NAT@spacechar\fi}%
       {\@citeb\@extra@b@citeb}%
     \NAT@date}}
  {\@citea\NAT@nmfmt{\NAT@nm}%
   \NAT@spacechar\NAT@@open\if*#1*\else#1\NAT@spacechar\fi\NAT@hyper@{\NAT@date}}
  {}{}
\makeatother

% Listings

% Verbatim

% Tables

% Graphics

% Make links footnotes instead of hotlinks:
%  \setlength{\emergencystretch}{3em}  % prevent overfull lines
 \providecommand{\tightlist}{%
   \setlength{\itemsep}{0pt}\setlength{\parskip}{0pt}}
   
% Numbered sections
\setcounter{secnumdepth}{5}
% % % Redefines (sub)paragraphs to behave more like sections
% \ifx\paragraph\undefined\else
% \let\oldparagraph\paragraph
% \renewcommand{\paragraph}[1]{\oldparagraph{#1}\mbox{}}
% \fi
% \ifx\subparagraph\undefined\else
% \let\oldsubparagraph\subparagraph
% \renewcommand{\subparagraph}[1]{\oldsubparagraph{#1}\mbox{}}
% \fi
% 
% Spacing
\doublespacing

% Title
\title{Pre-Analysis Plan:\\
Casting the First Stone: Understanding Attitudes towards Lynching in
Brazil\footnote{We thank Rosario Aguilar, Louise Araújo, Miriam Golden,
  Umberto Mignozzetti, Irfan Nooruddin, Catarina Roman, and Alex Scacco
  for their valuable comments. This pre-analysis plan received IRB
  approval from Brown University in October 2020 (Protocol 2009002803).}}

% Author
\author{Danilo Freire\footnote{Postdoctoral Research Associate, The Political
  Theory Project, Brown University, 8 Fones Alley, Providence, RI 02912,
  \href{mailto:danilofreire@gmail.com}{\texttt{danilofreire@gmail.com}},
  \url{https://danilofreire.github.io}.} \and David Skarbek\footnote{The Department of Political Science and the
  Political Theory Project, Brown University, 8 Fones Alley, Providence,
  RI 02912, USA,
  \href{mailto:david_skarbek@brown.edu}{\texttt{david\_skarbek@brown.edu}},
  \url{http://davidskarbek.com}.}}

% Date
\date{\today}

% Begin document
\begin{document}
\maketitle

% Abstract
\begin{abstract}
\doublespacing \noindent Why do citizens support extrajudicial violence? While endorsement for
vigilantism has significantly decreased in advanced democracies,
lynchings remain tolerated throughout the developing world. In this
pre-analysis plan, we propose three survey experiments to investigate
(i) which criminal profiles citizens prefer for extralegal punishment,
(ii) how individuals justify mob violence, and (iii) whether information
provision reduces support for vigilantism. First, we run a conjoint
experiment to identify criminal characteristics that are associated with
higher likelihood of lynching. In our second experiment, we show
respondents a short article about a real lynching episode and ask them
whether they see the mob's motivations as legitimate. Lastly, we test
whether providing information about the legal consequences of lynching,
human rights guarantees, and the risk of retribution makes respondents
less likely to endorse extralegal violence. We will run the experiments
in Brazil, a country which has seen a sharp rise in vigilante attacks
and currently experiences one lynching attempt per day. Our survey will
be conducted online and include 2,000 participants.
\vspace{.25cm}

\noindent \textbf{Keywords}: Brazil, crime, extralegal violence, lynching, vigilantism
\vspace{.25cm}

\end{abstract}


% Table of Contents
\newpage

\hypertarget{motivation}{%
\section{Motivation}\label{motivation}}

\label{sec:motivation}

\doublespacing

Why do citizens endorse extralegal punishment?\footnote{Here we consider
  extralegal violence, vigilantism, lynching, and related terms as
  synonyms. We follow \citet[6]{moncada2017varieties} and define
  vigilantism as ``the collective use or threat of extralegal violence
  in response to an alleged criminal act''. While we focus here on
  vigilantism that directly involves physical violence, we make no
  distinction between lethal and non-lethal lynching episodes.} Although
mob violence has decreased in advanced democracies, vigilantism remains
widespread in the developing world. In particular, Latin America has
been severely affected by a recent increase in lynchings. Motivated by
growing numbers of drug-related crime and by police ineffectiveness,
citizen violence has surged in the region \citep{mallen2014vigilantes}.
Brazil provides a telling example. From 2011 to 2015, the country
registered about 2,500 lynching episodes, and in 2015 alone 173 people
were killed by angry mobs, an average of one execution every two days
\citep{barbara2015vigilantes, oliveira2016mob}. According to José
Martins, who has studied lynchings in Brazil for more than 30 years,
those figures are not only the highest in the country's history, but
also the highest in the world \citep{pearson2018latam}. Victims are
generally accused of petty theft, and many are beaten to death for
stealing mobile phones, cooking utensils, or pairs of sandals
\citep{barbara2015vigilantes}. Lynching perpetrators are often young
men, but they also include female teenagers, elderly women, and even
members of local police forces
\citep{moura2017linchamentos, oliveira2016mob}.

Recent research suggests that support for vigilantism in Latin America
is associated with exposure to violence \citep{garcia2019anger},
untrustworthy law enforcement \citep{zizumbo2017community}, perceived
insecurity \citep{ceobanu2011crime, godoy2004justice}, and social
inequality \citep{arias2010violent}. However, approval for extrajudicial
violence raises important questions about the relationship between
states and societies in the region \citep{nivette2016institutional}.
First, while vigilantism provides social order to vulnerable
populations, it also undermines the legitimacy of state institutions
\citep{schuberth2013challenging}. Lynching episodes reinforce the idea
that civilians should not wait for legal proceedings and that social
control is better exerted by the people themselves
\citep{black1983crime}. Second, victim characteristics greatly impact
the likelihood and severity of lynchings. In Brazil, a person is rarely
lynched when they commit a crime against males, and females that commit
the same crimes as males also receive less punishment.\footnote{See
  \url{https://revistapesquisa.fapesp.br/en/mob-injustice}. Access:
  June, 2020.} Moreover, there is no significant social difference
between victims and perpetrators, nor are black Brazilians particularly
targeted by lynch mobs \citep{oliveira2016mob}. These dynamics pose a
puzzle to existing explanations for mob violence support and stand in
sharp contrast with the American experience
\citep{smaangs2016doing, wood2011lynching}.

In this study, we endeavour to answer the following questions: 1) which
victim characteristics increase individual support for extralegal
punishment?; 2) how do individuals justify their approval for
vigilantism?; and 3) can information provision reduce support for mob
violence? We design a set of three experiments to tackle these issues.
Our first experiment consists of a conjoint analysis in which we show
respondents different profiles of lynching victims and ask them who they
believe deserve punishment. In our second experiment, we test how
ineffective policing, slow criminal justice systems, and demand for
hasher legal punishment affect individual propensity to support mob
violence. Finally, we assess whether citizens become less supportive of
vigilantism when they are informed about the legal consequences of
lynchings, constitutional human rights, or the risk of \emph{vendettas}
associated with mob violence. While in this pre-analysis plan we present
the three experiments as described above, we will randomised their order
in our survey to mitigate potential order bias.

\hypertarget{experiment-01}{%
\section{Experiment 01}\label{experiment-01}}

\label{sub:exp01}

We start our analysis with a choice-based conjoint experimental design.
We will present five pairs of criminal profiles to respondents. Each
profile consists of eight attributes: 1) gender of the crime
perpetrator; 2) age of the crime perpetrator; 3) race of the crime
perpetrator; 4) residency of crime perpetrator; 5) offense; 6) gender of
the victim of the motivating crime; 7) age of the victim of the
motivating crime; 8) lynching perpetrators. The attributes and levels
are displayed in table \ref{tab:categories} below.

\vspace{.3cm}

\begin{table}[htpb]
\small
\begin{center}
\caption{\textbf{Attributes and Levels}}
\label{tab:categories}
\begin{tabular}{l !{\vrule width 1pt}p{9cm}}
\Xhline{2\arrayrulewidth}
\textbf{Attribute} & \multicolumn{1}{c}{\textbf{Levels}} \\
\Xhline{2\arrayrulewidth}
\small
Gender of crime perpetrator & Male; female \\ [4pt]
Age of crime perpetrator & Teenager; adult; elderly \\ [4pt]
Race of crime perpetrator & Black; White; Native Brazilian; Asian \\ [4pt]
Residency of crime perpetrator & Resident in the community; outsider \\ [4pt]
Offense & Picks the pocket; steals the car; molests; rapes; murders \\ [4pt]
Gender of crime victim & Male; female\\ [4pt]
Age of crime victim & Child; teenager; adult; elderly\\ [4pt]
Lynching perpetrators & Bystanders; neighbours; family of the victim; gangs; police \\
\Xhline{2\arrayrulewidth}
\end{tabular}
\end{center}
\end{table}

\newpage

Respondents have to indicate which profile they prefer for extrajudicial
punishment. We will also include a qualitative textbox below each
experiment for subjects to explain why they selected their case or
whether they believe neither alternative is appropriate.

Respondents will read the following prompt before they start the
experiment:

\begin{itemize}
\tightlist
\item
  Lynchings are often used as social punishment in Brazil. Lynchings are
  cases in which three or more people physically attack or execute a
  suspected criminal in public. We are interested in knowing more about
  how Brazilians see these episodes. In the next five questions, please
  read the description of two possible lynching victims in Brazil and
  indicate in which case you believe the punishment is more justified.
  Even if you are not entirely sure, please select one of the
  cases.\footnote{We will conduct the survey experiments in Portuguese.
    The translation for the first experiment is as follows: Linchamentos
    são às vezes usados como punição social no Brasil. Linchamentos são
    casos nos quais três ou mais pessoas agridem fisicamente ou executam
    em público um suspeito de um crime. Estamos interessados em saber
    mais sobre como os brasileiros vêem estes episódios. Nas próximas
    cinco questões, por favor, leia a descrição de duas possíveis
    vítimas de linchamento no Brasil e indique em quais delas você
    acredita que a punição é mais justificada. Mesmo que você não tenha
    certeza, por favor, escolha um dos casos.}
\end{itemize}

\hypertarget{sample}{%
\subsection{Sample}\label{sample}}

\label{sub:sample}

We invite respondents from all regions of Brazil to participate in our
experiment. Qualtrics will recruit 2,000 adult Brazilian citizens to
take part in our survey experiment.

We will also include 15 questions on demographics that we believe may
influence the results. The questions are: 1) the respondent's age, 2)
gender, 3) state of birth, 4) state they currently live in, 5)
ethnicity, 6) level of education, 7) religion (if any), 8) monthly
family income in minimum wages, 9) political orientation (left to
right), 10) support for death penalty, 11) whether they had been
victimised in the last year, 12) whether they believe crime has
worsened, 13) whether they believe militias improve law and order, 14)
trust in the police, 15) trust in the judicial system.

\hypertarget{estimation}{%
\subsection{Estimation}\label{estimation}}

\label{sub:estimation}

We will estimate our models with the \texttt{cregg} package
\citep{leeper2018cregg} for the R statistical language
\citep{rstats2019}. We will report marginal means and average marginal
component effects (AMCEs) in our main analysis. However,
\citet{leeper2018subgroup} show that AMCEs can be misleading in subgroup
comparisons as model results are sensitive to the choice of reference
categories in interactions. In contrast, marginal means provide a clear
description of quantities of interest, in our case preferences towards
lynching, while allowing for easy comparisons between groups of
respondents. So while we will include both AMCEs and marginal means
estimators in our article, we will conduct our subgroup analysis using
only the latter. We will cluster standard errors by respondent.

\hypertarget{hypotheses}{%
\subsection{Hypotheses}\label{hypotheses}}

\label{sec:hypotheses}

We expect that, on average, respondents are more likely to select
profiles with the following attributes levels:

\begin{enumerate}
\def\labelenumi{\arabic{enumi})}
\tightlist
\item
  Gender of crime perpetrator: male
\item
  Age of crime perpetrator: teenager
\item
  Race of crime perpetrator: blacks
\item
  Residency of crime perpetrator: outsider
\item
  Offense: murder
\item
  Gender of victim: female
\item
  Age of victim: child
\item
  Lynching perpetrators: the family of the victim
\end{enumerate}

We also expect that support for extralegal punishment will be lower for
females of any age, whites and Asians, pick pocketers, and if the victim
of the motivating crime is male. Additionally, we expect that
respondents see lynchings as less legitimate if they are carried out by
bystanders, gangs, or the police forces. We have no prior beliefs about
the size of each coefficient.

\hypertarget{heterogeneous-effects}{%
\subsection{Heterogeneous Effects}\label{heterogeneous-effects}}

\label{sub:het01}

We also intend to run subgroup analysis in our sample. We believe that
preferences towards lynching may vary according to age, gender, race,
level of education, political ideology, and previous victimisation. In
particular, we hypothesise that:

\begin{itemize}
\item
  Younger respondents believe that lynchings by the police are less
  justified.
\item
  Female respondents believe that lynchings against females are less
  justified.
\item
  Respondents of a particular race are less likely to choose individuals
  of the same race as lynching victims.
\item
  More educated individuals show less support for lynchings against
  petty criminals.
\item
  Political ideology has no effect on support for lynchings when the
  victim is female.
\item
  Right-wingers show stronger support for lynchings against males than
  left-wingers.
\item
  Respondents who were victimised in the last year are more likely to
  support lynchings by neighbours and the police.
\end{itemize}

\hypertarget{experiment-02}{%
\section{Experiment 02}\label{experiment-02}}

In our second experiment, we analyse how respondents justify their
preferences for extralegal violence. We assess the impact of three
factors that have been cited as major drivers of vigilante justice in
Brazil: 1) police ineffectiveness; 2) slow criminal justice; 3) demand
for harsher punishment for criminals. Below, we discuss them in further
detail.

Research shows that police ineffectiveness frequently appears as a
strong predictor for vigilantism
\citep{cruz2019determinants, garcia2019anger}. The direct result of the
weakness of police institutions is that citizens decide to take criminal
matters ``into their own hands'', thus persecuting and punishing the
criminals by themselves. A recent statistic indicates that the police
solve only 10\% of the homicides in Brazil, what lends support to the
link between weak law enforcement and lynchings
\citep{pearson2018latam}.

Another determinant of support for vigilante justice that is suggested
in the literature is the lack of trust in the justice system
\citep{godoy2004justice, smith2019contradictions}. This is often due to
long criminal proceedings, which cause significant anxiety for the
victims. In Brazil, the penal code allows the accused to appeal each
decision several times, so it can take decades before a criminal case is
closed \citep{sousa2005utilizaccao}. In this respect, citizens do not
believe that, even if the criminal is put to trial, he/she will be
punished in a timely matter. Note that although the police is part of
the criminal justice system, we separate the two institutions in our
experiment.

Lastly, we evaluate whether respondents think that the legal punishment
assigned to criminals is not proportional to the severity of their
crimes. In particular, we hope to gauge the demand for iron-fisted
criminal justice in Brazil. Although this treatment arm is related to
the previous ones, it addresses not the efficiency of the institutions,
but their legitimacy \citep{nivette2016institutional}. We believe that
the demand for harsher laws is on the rise in Brazil, and this may be a
major reason why citizens justify their support for mob violence. In
fact, Brazilians are often vocal about their preference for repressive
legal punishment. In a recent article in \emph{The Wall Street Journal},
a bar owner justified the lynching of the local thug who killed his son
by saying that ``\emph{even if he had been put behind bars for 100 years
it wouldn't have been enough to pay for all his crimes}''
\citep{pearson2018latam}. We hypothesise that many Brazilians also share
this view.

The experiment consists of three treatment conditions and one control
group. Respondents will read an excerpt of a news article describing a
real lynching case. We have slightly edited the original text so that
respondents have no prior knowledge of the crime.\footnote{The original
  article is available at the following address:
  \url{https://jr.jor.br/2020/05/01/homem-e-linchado-na-vila-progresso}.
  Access: August 2020.} The vignette for the control group includes no
information about the reasons behind the lynching. We will ask
respondents to show their level of support for mob violence using a
0-100 slider, where 0 means no support and 100 means full support.
Respondents in each of the three treatment arms will read the same
piece, but with one additional sentence explaining the motivations
behind the lynching. The vignettes are as follows:

\begin{itemize}
\tightlist
\item
  \emph{Control group}: A man was lynched last Friday in Jundiaí, São
  Paulo. According to the neighbours, he tried to break into a house but
  was immobilised and beaten by members of the community.\footnote{In
    Portuguese: Um homem foi linchado na última sexta-feira em Jundiaí,
    São Paulo. De acordo com vizinhos, ele tentou invadir uma residência
    mas foi imobilizado e agredido por membros da comunidade.}
\end{itemize}

\begin{itemize}
\tightlist
\item
  \emph{Treatment 01 - Police ineffectiveness}: A man was lynched last
  Friday in Jundiaí, São Paulo. According to the neighbours, he tried to
  break into a house but was immobilised and beaten by members of the
  community. \textbf{One of the residents who took part in the lynching
  said they had beaten the suspect because ``the police never patrols
  the area''}.\footnote{In Portuguese: Um homem foi linchado na última
    sexta-feira em Jundiaí, São Paulo. De acordo com vizinhos, ele
    tentou invadir uma residência mas foi imobilizado e agredido por
    membros da comunidade. \textbf{Um dos moradores envolvidos no
    linchamento disse que eles agrediram o suspeito porque ``a polícia
    nunca patrulha o local''.}}
\end{itemize}

\begin{itemize}
\tightlist
\item
  \emph{Treatment 02 - Criminal justice ineffectiveness}: A man was
  lynched last Friday in Jundiaí, São Paulo. According to the
  neighbours, he tried to break into a house but was immobilised and
  beaten by members of the community. \textbf{One of the residents who
  took part in the lynching said they had beaten the suspect because
  ``the judicial system is too slow and the perpetrator is on the street
  until the case is heard''}.\footnote{In Portuguese: Um homem foi
    linchado na última sexta-feira em Jundiaí, São Paulo. De acordo com
    vizinhos, ele tentou invadir uma residência mas foi imobilizado e
    agredido por membros da comunidade. \textbf{Um dos moradores
    envolvidos no linchamento disse que eles agrediram o suspeito porque
    ``a justiça é muito lenta e os criminosos ficam soltos até o
    julgamento''}.}
\end{itemize}

\begin{itemize}
\tightlist
\item
  \emph{Treatment 03 - Demand for harsher legal punishment}: A man was
  lynched last Friday in Jundiaí, São Paulo. According to the
  neighbours, he tried to break into a house but was immobilised and
  beaten by members of the community. \textbf{One of the residents who
  took part in the lynching said they had beaten the suspect because
  ``the judicial punishment is not harsh enough''}.\footnote{In
    Portuguese: Um homem foi linchado na última sexta-feira em Jundiaí,
    São Paulo. De acordo com vizinhos, ele tentou invadir uma residência
    mas foi imobilizado e agredido por membros da comunidade. \textbf{Um
    dos moradores envolvidos no linchamento disse que eles agrediram o
    suspeito porque ``a punição da justiça não é dura o suficiente''}.}
\end{itemize}

Before each vignette, respondents will read the following text:

\begin{itemize}
\tightlist
\item
  You will be shown a news article. Please read it carefully. After you
  read the article, we will ask you one question about it.\footnote{In
    Portuguese: Uma notícia será apresentada para você. Por favor, leia
    a notícia com atenção. Após você ler o artigo, faremos uma pergunta
    sobre ele.}
\end{itemize}

After the vignette, respondents will be presented with this question:

\begin{itemize}
\tightlist
\item
  Would you think that the lynching was justified? Please use the slider
  below to indicate your opinion. For disagreement, use 0 to 49; for
  agreement, use 51 to 100. Please use 50 if you neither agree nor
  disagree.\footnote{In Portuguese: Você acha que o linchamento foi
    correto? Por favor, use a barra abaixo para indicar sua opinião.
    Para discordar, use de 0 a 49; para concordar, use de 51 a 100. Por
    favor, use 50 para não concordar nem discordar.}
\end{itemize}

\hypertarget{sample-1}{%
\subsection{Sample}\label{sample-1}}

We will randomise the treatment and control conditions to the entire
respondent pool. The randomisation procedure here is independent of that
of the previous experiment.

\hypertarget{estimation-1}{%
\subsection{Estimation}\label{estimation-1}}

We will carry out our hypothesis tests with OLS. We will compare the
average score given by the control group with the average score
attributed by respondents in each treatment condition. We will use
robust standard errors for all the models.

\hypertarget{hypotheses-1}{%
\subsection{Hypotheses}\label{hypotheses-1}}

On average, we expect that:

\vspace{.5cm}

\noindent \emph{H1}: Respondents assigned to the \emph{police
ineffectiveness} treatment group will show stronger support for
lynchings when compared to the control group. We believe that the effect
will be smaller than for the other two treatment conditions.

\vspace{.3cm}

\noindent \emph{H2}: Respondents assigned to the \emph{criminal justice
ineffectiveness} treatment group will show stronger support for
lynchings when compared to the control group. We hypothesise that the
effect for this group will be higher than that of the first treatment
condition but lower than that of the third treatment group.

\vspace{.3cm}

\noindent \emph{H3}: Respondents assigned to the \emph{demand for
harsher punishment} treatment group will show the strongest support for
lynchings.

\hypertarget{heterogeneous-effects-1}{%
\subsection{Heterogeneous Effects}\label{heterogeneous-effects-1}}

We hypothesise that these respondents are more likely to agree that
lynchings are justified:

\begin{enumerate}
\def\labelenumi{\arabic{enumi})}
\tightlist
\item
  Older citizens
\item
  White
\item
  Low levels of education
\item
  Right-wingers
\item
  Individuals who had been previously victimised
\end{enumerate}

\hypertarget{experiment-03}{%
\section{Experiment 03}\label{experiment-03}}

Our last experiment aims to measure the effect of information provision
on attitudes about lynching. More specifically, we will test whether
reminding respondents about the legal and social consequences of
vigilante justice reduces the subjects' level of support for such
practice.

The experiment has three treatment conditions and a control group. In
all of them we will present respondents with a short statement affirming
that some Brazilians support vigilantism under certain conditions. The
control group will be asked to rate their agreement with the statement
with a 0-100 slider where 100 indicates full agreement with the
statement and 0 indicates no agreement. Respondents will be asked to use
0 to 49 if they disagree, 50 if they neither agree nor disagree, and
50-100 if they agree with the sentence.

Each of the three treatment groups will receive a different message
about the legal or social consequences of lynching in Brazil. In the
first treatment arm, we will inform subjects about how the Brazilian
constitution and penal code punishes civilian violence. The second
treatment group will be notified about the human rights guarantees
enshrined in Brazil's legal framework. The last group will read a
vignette that mentions how lynchings can spark \emph{vendettas} and
initiate a cycle of violence in the community. Subjects in the control
group will receive no information about the consequences of lynchings.
The text shown to the control and treatment groups can be read below.

\begin{itemize}
\tightlist
\item
  \emph{Control group}: In Brazil, some people believe that lynching may
  be justified under certain conditions. To what degree do you agree or
  disagree that lynching can be justified? Please use the slider below
  to indicate your preference. For disagreement, use 0 to 49; for
  agreement, use 51 to 100. Please use 50 if you neither agree nor
  disagree.\footnote{In Portuguese: No Brasil, algumas pessoas acreditam
    que linchamentos são justificados sob certas condições. O quanto
    você concorda ou discorda que linchamentos podem ser justificados?
    Por favor, use a barra abaixo para indicar sua preferência. Para
    indicar que discorda, use de 0 a 49; para concordar, use de 51 a
    100. Por favor, use 50 caso você não concorde nem discorde.}
\end{itemize}

\begin{itemize}
\tightlist
\item
  \emph{Treatment 01 - Legal punishment for lynching perpetrators}: In
  Brazil, some people believe that lynching may be justified under
  certain conditions. \textbf{However, the Brazilian constitution and
  penal code strictly forbid lynching and those involved can be accused
  of torture or murder}. To what degree do you agree or disagree that
  lynching can be justified? Please use the slider below to indicate
  your preference. For disagreement, use 0 to 49; for agreement, use 51
  to 100. Please use 50 if you neither agree nor disagree.\footnote{In
    Portuguese: No Brasil, algumas pessoas acreditam que linchamentos
    são justificados sob certas condições. \textbf{Entretanto, a
    constituição e o código penal do Brasil proíbem estritamente os
    linchamentos e os envolvidos podem ser acusados de tortura ou
    assassinato.} O quanto você concorda ou discorda que linchamentos
    podem ser justificados? Por favor, use a barra abaixo para indicar
    sua preferência. Para indicar que discorda, use de 0 a 49; para
    concordar, use de 51 a 100. Por favor, use 50 caso você não concorde
    nem discorde.}
\end{itemize}

\begin{itemize}
\tightlist
\item
  \emph{Treatment 02 - Human rights}: In Brazil, some people believe
  that lynching may be justified under certain conditions.
  \textbf{However, the Brazilian constitution states that all
  individuals have the right of not being tortured, including
  criminals}. To what degree do you agree or disagree that lynching can
  be justified? Please use the slider below to indicate your preference.
  For disagreement, use 0 to 49; for agreement, use 51 to 100. Please
  use 50 if you neither agree nor disagree.\footnote{In Portuguese: No
    Brasil, algumas pessoas acreditam que linchamentos são justificados
    sob certas condições. \textbf{Entretanto, a constituição do Brasil
    afirma que todos os indivíduos têm o direito de não serem
    torturados, inclusive criminosos}. O quanto você concorda ou
    discorda que linchamentos podem ser justificados? Por favor, use a
    barra abaixo para indicar sua preferência. Para indicar que
    discorda, use de 0 a 49; para concordar, use de 51 a 100. Por favor,
    use 50 caso você não concorde nem discorde.}
\end{itemize}

\begin{itemize}
\tightlist
\item
  \emph{Treatment 03 - Vendettas}: In Brazil, some people believe that
  lynching may be justified under certain conditions. \textbf{However,
  lynchings can trigger a new cycle of violence as the family or friends
  of the victim may retaliate the community}. To what degree do you
  agree or disagree that lynching can be justified? Please use the
  slider below to indicate your preference. For disagreement, use 0 to
  49; for agreement, use 51 to 100. Please use 50 if you neither agree
  nor disagree.\footnote{In Portuguese: No Brasil, algumas pessoas
    acreditam que linchamentos são justificados sob certas condições.
    \textbf{Entretanto, linchamentos podem iniciar um ciclo de violência
    pois a família ou amigos da vítima podem retaliar a comunidade}. O
    quanto você concorda ou discorda que linchamentos podem ser
    justificados? Por favor, use a barra abaixo para indicar sua
    preferência. Para indicar que discorda, use de 0 a 49; para
    concordar, use de 51 a 100. Por favor, use 50 caso você não concorde
    nem discorde.}
\end{itemize}

\hypertarget{sample-2}{%
\subsection{Sample}\label{sample-2}}

We will randomise the treatment and control conditions to all subjects
included in our sample. The treatment assignment here is independent of
that of the previous two experiments.

\hypertarget{estimation-2}{%
\subsection{Estimation}\label{estimation-2}}

We are interested in the difference between the average score assigned
by each of the treatment groups minus the average score assigned by the
control group. We will estimate average treatment effects using OLS with
dummy indicators for the treatment groups.

\hypertarget{hypotheses-2}{%
\subsection{Hypotheses}\label{hypotheses-2}}

In this experiment, we test the following hypotheses:

\vspace{.5cm}

\noindent \emph{H1}: Individuals who receive information about the
criminal consequences of lynching show lower support for vigilantism
that those in the control group.

\vspace{.3cm}

\noindent \emph{H2}: Reminding respondents that torture violates human
rights has no impact on their support for vigilantism.

\vspace{.3cm}

\noindent \emph{H3}: Information about \emph{vendettas} triggered by
lynchings makes individuals less likely to support vigilantism in
comparison with respondents in the control group. The effect of that
information is larger than the other two treatments.

\vspace{.5cm}

These three hypotheses allow us to test how respondents react to
different consequences of extralegal violence. Hopefully, the experiment
may also help policy-makers design more effective campaigns against
vigilantism in Brazil.

\hypertarget{heterogeneous-effects-2}{%
\subsection{Heterogeneous Effects}\label{heterogeneous-effects-2}}

As with the previous two experiments, we will measure whether our
treatments have heterogeneous effects. The same demographic
characteristics we describe above -- age, gender, race, education level,
political ideology, and previous victimisation -- will also be used
here.

\setlength{\parindent}{0cm}
\setlength{\parskip}{5pt}

% More bibliography
\bibliography{references.bib}

\end{document}
