\documentclass[a4paper,12pt]{article}
\usepackage{anysize}
\usepackage[T1]{fontenc}
\usepackage[stable]{footmisc}
\usepackage{setspace}
\usepackage{lmodern}
\usepackage{libertine}
\usepackage[libertine]{newtxmath}
\usepackage[scale=0.825]{FiraMono}
\usepackage[top=2cm,bottom=2cm,left=2cm,right=2cm]{geometry}
\usepackage{mathtools}
\usepackage[authoryear]{natbib}
\usepackage[UKenglish]{babel}
\usepackage[UKenglish]{isodate}
\usepackage{babelbib}
\usepackage{graphicx}
\usepackage{booktabs, makecell, longtable}
\usepackage{dcolumn}
\usepackage{float}
\usepackage[caption = false]{subfig}
\floatplacement{figure}{H}
\usepackage{caption}
\usepackage{rotating}
\usepackage{pdflscape}
\usepackage{pdflscape}
\usepackage{ifthen}
\usepackage{graphicx}
\usepackage[usenames,dvipsnames]{xcolor}
\definecolor{darkblue}{rgb}{0.0,0.0,0.55}
\usepackage{tikz}
\usetikzlibrary{shapes.geometric, arrows}
\setcitestyle{aysep={}}
\usepackage{etoolbox}
\makeatletter
\patchcmd{\NAT@citex}
  {\@citea\NAT@hyper@{%
	 \NAT@nmfmt{\NAT@nm}%
	 \hyper@natlinkbreak{\NAT@aysep\NAT@spacechar}{\@citeb\@extra@b@citeb}%
	 \NAT@date}}
  {\@citea\NAT@nmfmt{\NAT@nm}%
   \NAT@aysep\NAT@spacechar\NAT@hyper@{\NAT@date}}{}{}
\patchcmd{\NAT@citex}
  {\@citea\NAT@hyper@{%
	 \NAT@nmfmt{\NAT@nm}%
	 \hyper@natlinkbreak{\NAT@spacechar\NAT@@open\if*#1*\else#1\NAT@spacechar\fi}%
   {\@citeb\@extra@b@citeb}%
	 \NAT@date}}
  {\@citea\NAT@nmfmt{\NAT@nm}%
   \NAT@spacechar\NAT@@open\if*#1*\else#1\NAT@spacechar\fi\NAT@hyper@{\NAT@date}}
  {}{}
\makeatother
\cleanlookdateon
\exhyphenpenalty=1000
\hyphenpenalty=1000
\widowpenalty=1000
\clubpenalty=1000
\usepackage{hyperref}

\hypersetup{
	breaklinks=true,
	linkcolor=Mahogany,
	citecolor=Mahogany,
	urlcolor=darkblue,
	colorlinks=true}

\newcommand*\justify{%
  \fontdimen2\font=0.4em% interword space
  \fontdimen3\font=0.2em% interword stretch
  \fontdimen4\font=0.1em% interword shrink
  \fontdimen7\font=0.1em% extra space
  \hyphenchar\font=`\-% allowing hyphenation
}

\begin{document}

\doublespacing

\noindent \textbf{Research \& Politics}

\noindent December 12, 2022

\vspace{.5cm}

\noindent Dear Editor and Reviewers,

\vspace{.5cm}

Thank you for giving us the opportunity to revise our manuscript, ``Vigilantism
and Institutions: Understanding Attitudes toward Lynching in Brazil'' (Ms. No.
RAP-22-0164). We have made several changes to the article based on the feedback
from the editor and reviewers, and we believe the manuscript has improved
significantly as a result.

In particular, we have focused on addressing the scope and methodology of our
study, as highlighted by Reviewers 1 and 2. We have edited the Introduction and
added a new paragraph to the Conclusion to better explain the scope of our
work. We have also included more details about our methodology in section two
and in the Supplementary Materials. In response to Reviewer 2's suggestion, we
have removed the significance tests from the manuscript. We have also added a
footnote to the Supplementary Materials indicating that we randomized the order
of the experiments to reduce the risk of carryover effects.

Furthermore, we have added a discussion of how our findings compare with those
of other studies, as suggested by Reviewer 1. We have also included the results
of our subgroup analyses in the Supplementary Materials, again following
Reviewer 1's recommendation. Additionally, we have included all references
suggested by Reviewers 1 and 2 in the main body of the manuscript.

Overall, we believe that the manuscript is now stronger and more appealing to a
broader audience. We discuss the changes in more detail below. We are grateful
for the feedback and suggestions provided by the editor and reviewers, and we
hope that we have addressed all of their concerns. Thank you again for the
opportunity to revise our manuscript.

\vspace{.5cm}

\noindent Sincerely Yours,

\vspace{.5cm}

\noindent The Authors

\newpage

\section*{Editor Comments and Responses}

\textbf{1)} The editor writes: ``I agree with R2 that the framing of the study
could be improved. You currently cite a number of interesting figures and basic
facts about lynching in Brazil in the start of the conclusion. I think this
would fit better in the introduction. R2 also argues that the reference to the
Blair et al study on greater costs of local violence is imprecise. I am
guessing you mean something along the lines of likely to affect a greater
number directly than terrorism or civil war, which I think is reasonable but
could be said more directly.''

\vspace{.3cm}

\noindent \textbf{Response:} 
\begin{quote}

We have moved the facts about lynchings in Brazil to the Introduction and 
the section now provides more context for our experiments. The
second paragraph of the Introduction now reads as follows (page 02):
``\textit{One of the most serious forms of extrajudicial violence is lynching.
  Lynching can be defined as ``incidents of physical violence committed by
  large numbers of private citizens against one or more individuals accused of
  having committed a ``criminal'' offense, whether or not this violence
  resulted in the death of the victim(s)'' \citep[645]{godoy2004justice}.
  Although lynchings occur in more than one hundred countries in all regions of
  the world \citep{jung2020lynching,smith2019contradictions}, Latin America has
  been particularly affected by a sharp increase in vigilante violence.
  Lynching episodes have been reported in Guatemala, Mexico, Peru, Venezuela
  and other countries in the region \citep{barbara2015vigilantes,
  cruz2019determinants, godoy2004justice}. From 2011 to 2015, Brazil registered
  about 2,500 lynching episodes, and 173 people were killed by angry mobs in
  2015 alone---nearly one execution every two days
  \citep{barbara2015vigilantes, oliveira2016mob}. According to José de Souza
  Martins \citeyearpar{martins2015linchamentos}, who has studied lynchings in
  Brazil for more than thirty years, these figures are not only the highest in
  the country's history, but among the highest in the world. The people who
participate in lynchings are typically young men, but they also sometimes
include teenage women and girls, elderly women, and even members of the local
police \citep{moura2017linchamentos}.}''

We have also changed the last sentence in the first paragraph to make our
claims more precise. It now reads (page 02): ``\textit{This
omission is significant not only because vigilantism can deepen group enmities
and lead to cycles of violence, but also because it undermines the legitimacy
of the state and the rule of law, which are essential for economic development
and democratic stability \citep{jung2020lynching, tankebe2009self}}.''

\end{quote}

\vspace{.3cm}

\noindent \textbf{2)} The editor writes: ``The manuscript is focused primarily
on the empirical approach, but I agree with R1 that you could say a bit more
about the rationale for the covariates considered, perhaps drawing out why more
``novel'' features such as family or type of crime is likely to be important.'' 

\vspace{.3cm}

\noindent \textbf{Response:} 
\begin{quote}

We have followed Reviewer 1's suggestion and have extended section two (``When
Is Lynching Perceived as More Justified?'') to provide further information
about our choice of covariates. The third paragraph of that section now reads
(page 04): ``\textit{In addition to our knowledge of the Brazilian case, we
  also chose our attributes based partly on existing work on crime,
  vigilantism, and extrajudicial violence. From the prison violence literature,
  we know that the relative age differences and the genders of perpetrators and
  victims affect how it is perceived \citep{fleisher2009myth}. We likewise know
  that certain offenses are seen as especially reprehensible
  \citep{skarbek2014social}, so we included such offenses, like molestation,
  alongside less serious offenses. We include race because, in the United
  States, violent lynchings have often been wielded for racist reasons
  \citep{dray2003hands}. Likewise, we include residency because extrajudicial
  violence is often used against people ``who don't belong'' or are
  ``outsiders'' in some respect. Finally, in honor cultures that value
  retaliatory violence, people believe that victims and the family of victims
  have a special right (and often an obligation) to enact retribution
\citep{weiner2013rule}.}

As we show in the next section, we have also included a new paragraph in the
Conclusion to discuss the relationship between our findings and the existing
literature on extralegal violence. By doing this, we aim to provide a more
comprehensive overview of our work and its significance within the broader
field.

\end{quote}

\vspace{.3cm}

\noindent \textbf{3)} The editor writes: ``On the presentation of the results,
I agree with R2 that the discussion of the results does not need to re-report
all the significance tests, and the full tables could be included in the text
or appendix.''

\vspace{.3cm}

\noindent \textbf{Response:} 
\begin{quote}

We have deleted the statistical details from the main text to improve the
readability of the manuscript. All significance tests are now reported only in
the Supplementary Materials.

\end{quote}

\vspace{.3cm}

\noindent \textbf{4)} The editor writes: ``It would be useful to add a bit more
discussion of Brazil in comparative perspective. I wonder if the results for
Brazil are likely to be more general and that the US could be an outlier, for
example.''

\vspace{.3cm}

\noindent \textbf{Response:} 
\begin{quote}

We have made the requested changes and significantly expanded the Conclusion
section based on the feedback from the editor and Reviewer 1. We are grateful
for this suggestion, as it has allowed us to better connect our article to the
broader literature on vigilantism.

The third paragraph of the Conclusion, now the longest in the section, is
exclusively dedicated to discussing the scope conditions of our research. We
have linked our findings to important works in the literature and have also
discussed how a culture of honor may help to explain some of our main findings.
This new paragraph provides valuable insight and enhances the broader
significance of our study. Overall, we believe that the revised Conclusion
section is stronger and more useful to readers interested in this topic.

The paragraph reads (page 11): ``\textit{These findings have implications that
  extend well beyond Brazil. In particular, our experiments indicate that
  lynchings have a crucial gendered aspect, which has also been documented in
  other contexts. Using data from 18 Latin American countries,
  \citet{nivette2016institutional} also finds that respondents are most likely
  to support lynchings when the criminal raped a child, and studies about
  lynchings in the American South point out that several episodes resulted from
  accusations of sexual assault \citep{jacquet2013giles, smaangs2020race}. We
  believe that a culture of honor may explain these results. Individuals in
  honor societies view crime as an attack on their personal reputation and, in
  turn, are more likely to take revenge to defend their status and that of
  people perceived as deserving protection, such as women and children
  \citep{nisbett2018culture}. A culture of honor may also explain why
  Brazilians see lynchings carried out by the family of the victim as more
  justified, as well as refraining from using extralegal violence if it can
  trigger vendettas. We expect similar results in societies which share those
  cultural norms. Lastly, the fact that race does not appear to be a major
  motivation behind lynchings also reflects the experience of places like Haiti
  or Southern Africa, where popular violence was mainly driven by other social
  factors \citep{berg2011globalizing, jung2020lynching}. Even in the American
  South, Whites and Blacks also lynched people of their own race
  \citep{beck1997race}. In this respect, our paper also highlights that
lynchings in the Global South may be more strongly connected with the idea of
``popular justice'' than with racial animus \citep{martins2015linchamentos}.}''

\end{quote}

\vspace{.3cm}

\noindent \textbf{5)} The editor writes: ``Finally, in order to ensure any
delays on a decision on the resubmission I would encourage you to make sure
that the final manuscript follows the journal guidelines,
\url{https://journals.sagepub.com/author-instructions/rap}. For example, the
sections should not be numbered, and the references should have first initial
only rather than full names and books should have location before publisher
(please ensure that all references are complete – Grimm looks incomplete to me,
for example).''

\noindent \textbf{Response:} 
\begin{quote}

The sections are now unnumbered and we have completed the references according
to Research \& Politics' guidelines.

\end{quote}

\vspace{.3cm}

We would like to thank the editor Kristian Skrede Gleditsch for his very
helpful comments and suggestions.

\section*{Reviewer 1 Comments and Responses}

\noindent \textbf{1)} Reviewer 1 writes: ``The empirical approach is presented
as offering many benefits---I would also acknowledge the drawbacks. And many of
these results do not seem to square with other parts of the world (which is a
feature and I would ask the authors for a little bit of speculation as to why!)
but I would suggest the authors think about the scope conditions imposed by the
setting and how well they seem to extend (the gendered aspect, yes, the race
aspect, perhaps there are some conditions). The finding I think is most
interesting to draw out is the family of rape/murder victims, this is—I think—a
facet not typically focused on and strikes me as the ``newest'' finding and one
I would encourage the authors to pull forward and (in the context of a strict
word limit) emphasize and do a bit of theoretical speculation for others to
build upon!''

\vspace{.3cm}

\noindent \textbf{Response:} 

\begin{quote}
 
Reviewer 1 raises an important issue. We agree that we had not adequately
explained the scope conditions of our research and have therefore added a new
paragraph to the Conclusion to provide additional context. In that paragraph,
we compare our results with studies on vigilantism in other Latin American
countries and the United States, and we also discuss the role of social norms,
particularly those related to a culture of honor, in shaping attitudes towards
lynching in Brazil. Additionally, we speculate on why race does not seem to
influence individual preferences for lynchings in Brazil.

The paragraph reads (page 11): ``\textit{These findings have implications that
  extend well beyond Brazil. In particular, our experiments indicate that
  lynchings have a crucial gendered aspect, which has also been documented in
  other contexts. Using data from 18 Latin American countries,
  \citet{nivette2016institutional} also finds that respondents are most likely
  to support lynchings when the criminal raped a child, and studies about
  lynchings in the American South point out that several episodes resulted from
  accusations of sexual assault \citep{jacquet2013giles, smaangs2020race}. We
  believe that a culture of honor may explain these results. Individuals in
  honor societies view crime as an attack on their personal reputation and, in
  turn, are more likely to take revenge to defend their status and that of
  people perceived as deserving protection, such as women and children
  \citep{nisbett2018culture}. A culture of honor may also explain why
  Brazilians see lynchings carried out by the family of the victim as more
  justified, as well as refraining from using extralegal violence if it can
  trigger vendettas. We expect similar results in societies which share those
  cultural norms. Lastly, the fact that race does not appear to be a major
  motivation behind lynchings also reflects the experience of places like Haiti
  or Southern Africa, where popular violence was mainly driven by other social
  factors \citep{berg2011globalizing, jung2020lynching}. Even in the American
  South, Whites and Blacks also lynched people of their own race
  \citep{beck1997race}. In this respect, our paper also highlights that
lynchings in the Global South may be more strongly connected with the idea of
``popular justice'' than with racial animus \citep{martins2015linchamentos}.}''

We have added three sentences on pages 6 and 7 to offer readers more context
about our covariates. They read (pages 6--7): ``\textit{Taken together, the
  results provide experimental evidence that support for lynchings in Brazil
  does not resemble the typical racial patterns scholars have observed in the
  United States \citep{dray2003hands,obert2018keeping,seguin2019national}.
  These findings are consistent with recent research on vigilantism in Haiti
  \citep{jung2020lynching} and with journalistic observations. They further
  suggest that the US experience with lynchings might be distinct from
vigilantism in other places and times \citep{oliveira2016mob}.}'' 

Finally, we have added two sentences on page 8 to discuss how cultural norms
are relevant to our results. They read (page 8): ``\textit{In sum,
respondents did not believe that lynchings should be carried out by the state
but did believe they should be used as a tool for individual or family
retribution. These results are consistent with norms of an honor culture in
which offenses are seen to tarnish the victim's status and the only way to
remove the stigma is through self-help efforts in the form of retaliatory
violence \citep{nisbett2018culture}.}''

\end{quote}

\vspace{.3cm}

\noindent \textbf{2)} Reviewer 1 writes: ``With respect to experiment two---I
would be very clear about the control group (in my reading) having been
``treated'' by the conjoint. In terms of estimation this is fine, but in terms
of baseline attitudes that we might use to benchmark in future studies, I do
worry a bit.  I would also encourage a bit more description of the subgroup
analyses on experiment 2 since I think they are likely interesting/important
for future work.''

\vspace{.3cm}

\noindent \textbf{Response:} 
\begin{quote}

We agree that our explanations regarding the experimental design were
inadequate. We actually randomized the order of the experiments to avoid
carryover effects, but we did not mention this explicitly in the text. We have
therefore added a footnote to the Supplementary Material to clarify this point.
The footnote, which can be found on page 83 of the SM, reads: ``\textit{To
prevent potential carryover effects caused by the conjoint, we randomized the
order of the conjoint and the information provision experiments
\citep{perreault1975controlling}.''} We believe that this procedure ensures
that our estimates are unbiased.

We agree that the subgroup analyses may be of interest to scholars. In response
to this feedback, we have added a short comment about each of our subgroup
tests to the Supplementary Materials. These comments provide information about
the statistical significance of the results and highlight any variations that
we observed according to the covariates that we controlled for. For example, in
Section C.4.1 of the Supplementary Materials, where we assess whether our
results for the conjoint experiment differ between genders, we write the
following (page 24): ``\textit{Results do not seem to vary according to the
  gender of the respondent. We focus here on the differences between males and
  females and exclude the 11 observations in which respondents preferred not to
say their gender or marked ``other'' in our questionnaire. Across all conjoint
experiment attributes, we see an overlap between the 95\% confidence intervals
for males and females.}'' Similarly, in Section C.4.3, in which we test whether
our results vary by race, we write (page 32): ``\textit{Below are our results
  when we disaggregate the data by race. We find that they are almost identical
  is all dimensions except for offense. Asian respondents are much less likely
to select profiles that contain pickpocketing as a crime.}''  

All analyses are also accompanied by their corresponding \texttt{R} code to
facilitate the replication of our results. 

\end{quote}

\vspace{.3cm}

\noindent \textbf{3)} Reviewer 1 writes: ``Minor: 1) I recommend citing Nick
Rush Smith and there is a new special issue out at Comparative Politics that
looks to have several pieces that are relevant! 2) I would like to see the
supplementary materials included!''

\vspace{.3cm}

\noindent \textbf{Response:} 
\begin{quote}

We have added a citation to Nick Rush Smith's work on page 02 and a citation to
the special issue in footnote 01 (page 02). The Supplementary Materials are now
included in the submission as well.

\end{quote}

\vspace{.3cm}

We would like to thank Reviewer 1 for her/his helpful comments and suggestions.

\section*{Reviewer 2 Comments and Responses}

\textbf{Major issues:}

\noindent \textbf{1)} Reviewer 2 writes: ``Framing. The abstract is rather
short, and I believe R\&P abstracts can be up to 200 words. It would be good to
make use of the space to highlight additional results that are especially
counterintuitive or interesting.''

\vspace{.3cm}

\noindent \textbf{Response:} 
\begin{quote}

We agree that the abstract could be improved. We have added two sentences to
the abstract and it now reads as follows: ``\textit{Why do people support
  extrajudicial violence? In two survey experiments with respondents in Brazil,
  we examine which characteristics of lynching scenarios garner greater support
  for lynching and whether providing different types of information about
  lynching reduces support for it. We find that people often do support
  community members to take vengeance. In particular, our analysis finds that
  people strongly support the use of extrajudicial violence by families of
  victims against men who sexually assault and murder women and children. We
  also find that criminal punishment and the threat of vendettas reduce
  support, but appeals to the human rights of victims have zero effect on
  support for lynchings. Unlike the U.S. experience with lynchings, race was
not observed to play an important role in how respondents answered the
survey.}''

\end{quote}

\vspace{.3cm}

\noindent \textbf{2)} Reviewer 2 writes: ``The introduction motivates the study
by noting that ``local violence'' can be more costly than war or terrorism
violence. This raises questions about what is ``local violence'', and I'm not
sure the cited source after this statement clarifies the situation. The cited
Blair at al piece indicates that local violence is ``possibly'' more costly
than war or terrorism, but this seems imprecise to me. If there is more space
to defend this statement more, that might be ok, but it is probably better just
to cut it and motivate the study in other ways.''

\vspace{.3cm}

\noindent \textbf{Response:} 
\begin{quote}

Reviewer 2's concern about the imprecision of our statement is valid. We have
deleted the original sentence and replaced it with the following (page 02):
``\textit{This omission is significant not only because vigilantism can
  entrench group animosities and create vicious cycles of violence, but also
because it severely undermines state legitimacy and the rule of law, two key
pillars of democratic stability and economic development
\citep{jung2020lynching, tankebe2009self}}.''

\end{quote}

\vspace{.3cm}

\noindent \textbf{3)} Reviewer 2 writes: ``Why is the scale of disagree to
agree 0-100? Could the authors cite a source or sources on why this precise
range was used, and briefly discuss some tradeoffs with such a scale? Why not
a, for example, seven-point Likert scale?''

\vspace{.3cm}

\noindent \textbf{Response:} 
\begin{quote}

Reviewer 2 is correct that we did not explain why we used a 0-100 scale in our
study. We chose this scale because it allows us to measure a wider range of
responses and capture the nuances of our treatments more accurately. This scale
is also easy to interpret and convert into discrete categories, which makes it
useful for our analysis. Additionally, the 0-100 scale is the default range
used by many survey platforms, such as
\href{https://www.surveymonkey.co.uk/curiosity/new-introducing-slider-star-rating-question-types/}{SurveyMonkey},
\href{https://www.questionpro.com/blog/use-slider-scales-for-a-more-accurate-rating/}{QuestionPro},
and
\href{https://www.qualtrics.com/support/survey-platform/survey-module/editing-questions/question-types-guide/standard-content/slider/}{Qualtrics},
which was the firm we used to conduct our survey experiments. As such, our
respondents are likely familiar with this scale.

We implemented the 0-100 scale following the recommendation of
\citet{carter2020}, who used the same range in their research design. While we
agree that a seven-point Likert scale would also be a good option, we believe
that the 0-100 scale is appropriate for our purposes. 

\end{quote}

\vspace{.3cm}

\noindent \textbf{4)} Reviewer 2 writes: ``Related to the above, the wording of
the instructions seem to suggest that respondents could select anywhere between
0 and 49 if they disagree; unless I am mistaken, the instructions do not
indicate, for example, that 0 indicates fully disagree and 100 indicates fully
agree. Were any more detailed instructions provided? Is there any evidence that
people fully understood the instructions in this regard?''

\vspace{.3cm}

\noindent \textbf{Response:} 
\begin{quote}

Reviewer 2 raises a valid concern about the instruction for our second
experiment. While we acknowledge that we could have worded the instructions
better, we believe that respondents were able to understand them as intended.
As we mentioned earlier, Qualtrics uses the 0-100 scale by default in its
survey platform, so our respondents, who were recruited from Qualtrics' online
panel, are likely familiar with this scale. Additionally, we have plotted the
distribution of responses for the second experiment and have found that the
majority of respondents selected zero, which indicates that they do not agree
that lynchings are justified. This is consistent with the findings of our
quantitative text analysis, which estimates a latent Dirichlet allocation (LDA)
model and finds that one of the most common topics in the text responses is
``lynchings are not justified'' (page 81 of the Supplementary Materials).

\begin{figure}
  \centering
  \includegraphics[width=0.8\textwidth]{hist.pdf}
  \caption{Support for Lynchings - Experiment 2}
  \label{fig:hist}
\end{figure}

We then cross-checked the distribution of responses with the qualitative
evidence we collected for the conjoint experiment, in which participants had to
justify their choices. When we filter our data to only include respondents who
selected zero on the 0-100 scale, we find the following justifications for
their choices (Portuguese version in parentheses): 

\begin{itemize}

  \item ``\textit{It is not justified, but I am forced to express my opinion. We have a death here, but it is not justified}.'' (``não se justifica, mas sou obrigado a opinar. temos uma morte. mas não se jutifica'') 
  \item ``\textit{I only chose it because I had to select a case, as stated. But I do not agree with either of the lynchings and I believe that justice should cover the gap that people use as an excuse for such practice}.'' (``escolhi somente pois deveria obrigatoriamente selecionar um caso, como enunciado. mas não concordo com nenhum dos dois linchamentos e acredito que a justiça deveria cobrir a lacuna que as pessoas utilizam como desculpa para tal prática'')
  \item ``\textit{I don't know... I selected one because I had to do so, but I am strongly against lynching. Eye for an eye, tooth for a tooth}.'' (``não sei bem... escolhi porque tinha que escolher, mas sou fortemente contra linchamento. olho por olho, dente por dente'')
  \item ``\textit{First of all, I want to make it clear that I do not condone either of the two cases. I think there should be no lynching, regardless of the crime. In this approach, I think case 2 is much more serious, because the victim is a child and the perpetrator is an adult, unlike in case 1}.'' (``à princípio quero deixar claro que não compactuo com nenhum dos dois casos. acho que não deve haver linchamento, independente do crime. nessa abordagem, acho que o caso 2 é bem mais grave, pois a vítima se trata de uma criança e autor de adulto, diferentemente do caso 1'')
  \item ``\textit{Either of the answers is immoral. Never, under any circumstances, is vigilante justice justified}.'' (``qualquer das respostas é imoral. nunca, em hipótese alguma a justiça pelas próprias mãos é justificada'')

\end{itemize}

There are several more such examples in the text. In contrast, when we include
only individuals who selected 100 on the scale, these are some of the
justifications mentioned in the first experiment:

\begin{itemize}
  \item ``\textit{A killer deserves the death penalty}.'' (``um assassino merece pena de morte.'')
  \item ``\textit{Because it is a murder: eye for an eye!}'' (``por tratar-se de um assassinato: olho por olho !'')
  \item ``\textit{There must be some punishment}.'' (``tem que haver punição.'')
  \item ``\textit{In both cases lynchings is appropriate}.'' (``nos dois casos cabe o linchamento.'')
  \item ``\textit{Because Brazil's Judiciary is s**t}.'' (``porque a justiça do brasil é uma merda'')
  \item ``\textit{Even if the criminal is indigenous, they must pay, you do not mess with a child}.'' (``mesmo que o criminoso seja indigena, tem que pagar, com criança não se mexe'')
\end{itemize}

All sentences above are included in the \texttt{q13\_text} variable in our
data. Although we cannot provide a statistical test to assess whether all
participants understood the instructions correctly, the evidence above suggests
that respondents did not find the scale confusing and that the quantitative
responses largely mirror our qualitative evidence. We appreciate Reviewer 2's
feedback and are grateful for the opportunity to address this concern.

\end{quote}

\vspace{.3cm}

\noindent \textbf{5)} Reviewer 2 writes: ``In the results section, it is
probably not necessary to list the coefficient, SE, and p-value in the text for
so many variables when the figures show the most important information,
assuming the full tables are in the supplementary information.'' 

\vspace{.3cm}

\noindent \textbf{Response:} 
\begin{quote}

We agree that it is not necessary to list the significance tests in the text.
We have removed them from the manuscript as suggested.

\end{quote}

\vspace{.3cm}

\noindent \textbf{6)} Reviewer 2 writes: ``There is other recent survey work on
support for vigilantism that could be acknowledged, like Zizumbo-Colunga 2017.
On Brazil, there is Schuberth and others. Beyond survey work, there is a
growing line of work on vigilantes or autodefensas in other Latin American
countries, such as Peru or Mexico, that could be briefly discussed. For example
Kloppe-Santamaria on Puebla in JLAS or Trevizo on Mexico generally in LAPS.
This work seems relevant in many ways.''

\vspace{.3cm}

\noindent \textbf{Response:} 
\begin{quote}

The reviewer is correct that there is a growing literature on vigilantism in
Latin American countries and the articles cited by the reviewer are indeed
relevant. We have added some of them to footnote number 01 (page 02), which now
reads: ``\textit{On vigilantism more generally, see
\citet{cohen2022collective}, \citet{schuberth2013challenging},
\citet{smith2019contradictions}, and \citet{zizumbo2017community}}.'' 

Paragraphs three and four of the Conclusion (pages 11--12), in turn, now read:
``\textit{Even in the American South, Whites and Blacks also lynched people of
  their own race \citep{beck1997race}. In this respect, our paper also
  highlights that lynchings in the Global South may be more strongly connected
with the idea of ``popular justice'' than with racial animus
\citep{kloppe2019lynching, martins2015linchamentos}.} [...] \textit{Finally,
lynchings have often been described as a response to low state capacity
\citep{trevizo2022mexico}, yet in some cases state agents actively incite or
engage in vigilante violence themselves \citep{arias2010violent}.}''

\end{quote}

\vspace{.3cm}

\noindent \textbf{7)} Reviewer 2 writes: ``Very minor note: on page 12,
significantly effect should be significantly affect.''

\vspace{.3cm}

\noindent \textbf{Response:} 
\begin{quote}
    We have corrected this typo in the revised version of the paper.
\end{quote}

\vspace{.3cm}

We would like to thank Reviewer 2 for her/his helpful comments and suggestions.

\bibliography{../article/references.bib}
\bibliographystyle{apalike}

\end{document}
