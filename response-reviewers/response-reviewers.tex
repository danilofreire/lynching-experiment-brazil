\documentclass[a4paper,12pt]{article}
\usepackage{anysize}
\usepackage[T1]{fontenc}
\usepackage[stable]{footmisc}
\usepackage{setspace}
\usepackage{lmodern}
\usepackage{libertine}
\usepackage[libertine]{newtxmath}
\usepackage[scale=0.825]{FiraMono}
\usepackage[top=2cm,bottom=2cm,left=2cm,right=2cm]{geometry}
\usepackage{mathtools}
\usepackage[authoryear]{natbib}
\usepackage[UKenglish]{babel}
\usepackage[UKenglish]{isodate}
\usepackage{babelbib}
\usepackage{graphicx}
\usepackage{booktabs, makecell, longtable}
\usepackage{dcolumn}
\usepackage{float}
\usepackage[caption = false]{subfig}
\floatplacement{figure}{H}
\usepackage{caption}
\usepackage{rotating}
\usepackage{pdflscape}
\usepackage{pdflscape}
\usepackage{ifthen}
\usepackage{graphicx}
\usepackage[usenames,dvipsnames]{xcolor}
\definecolor{darkblue}{rgb}{0.0,0.0,0.55}
\usepackage{tikz}
\usetikzlibrary{shapes.geometric, arrows}
\setcitestyle{aysep={}}
\usepackage{etoolbox}
\makeatletter
\patchcmd{\NAT@citex}
  {\@citea\NAT@hyper@{%
	 \NAT@nmfmt{\NAT@nm}%
	 \hyper@natlinkbreak{\NAT@aysep\NAT@spacechar}{\@citeb\@extra@b@citeb}%
	 \NAT@date}}
  {\@citea\NAT@nmfmt{\NAT@nm}%
   \NAT@aysep\NAT@spacechar\NAT@hyper@{\NAT@date}}{}{}
\patchcmd{\NAT@citex}
  {\@citea\NAT@hyper@{%
	 \NAT@nmfmt{\NAT@nm}%
	 \hyper@natlinkbreak{\NAT@spacechar\NAT@@open\if*#1*\else#1\NAT@spacechar\fi}%
   {\@citeb\@extra@b@citeb}%
	 \NAT@date}}
  {\@citea\NAT@nmfmt{\NAT@nm}%
   \NAT@spacechar\NAT@@open\if*#1*\else#1\NAT@spacechar\fi\NAT@hyper@{\NAT@date}}
  {}{}
\makeatother
\cleanlookdateon
\exhyphenpenalty=1000
\hyphenpenalty=1000
\widowpenalty=1000
\clubpenalty=1000
\usepackage{hyperref}

\hypersetup{
	breaklinks=true,
	linkcolor=Mahogany,
	citecolor=Mahogany,
	urlcolor=darkblue,
	colorlinks=true}

\newcommand*\justify{%
  \fontdimen2\font=0.4em% interword space
  \fontdimen3\font=0.2em% interword stretch
  \fontdimen4\font=0.1em% interword shrink
  \fontdimen7\font=0.1em% extra space
  \hyphenchar\font=`\-% allowing hyphenation
}

\begin{document}

\doublespacing

\noindent \textbf{Research \& Politics}

\noindent \today 

\vspace{.5cm}

\noindent Dear Editor and Reviewers,

\vspace{.5cm}

We would like to thank you for the opportunity to revise our manuscript,
``Vigilantism and Institutions: Understanding Attitudes toward Lynching in
Brazil'' (Ms. No. RAP-22-0164). Following the comments and suggestions of the
editors and reviewers, we have thoroughly revised our article and we believe it
has improved significantly as a result. 

We were especially committed to discussing the scope conditions of our article
and to clarifying some technical questions about our experiments as highlighted
by Reviewers 1 and 2. We did this by editing the Introduction, adding a new
paragraph to the Conclusion, and including more details about our methodology
in section two and in the Supplementary Materials. We also removed the
significance tests from the manuscript, as per the suggestion of Reviewer 2,
and added a footnote to the Supplementary Materials indicating that we
randomized the order of the experiments to reduce the risk of carryover
effects. Moreover, we now show how our findings compare with those found in
other studies, as suggested by Reviewer 1. We also discuss the results of our
subgroup analyses in the Supplementary Materials, again following the
recommendation of Reviewer 1. Lastly, we include all references suggested by
Reviewer 1 and 2 in the main body of the manuscript. We are extremely
appreciative of these recommendations and agree that the manuscript is now
stronger and appeals to a broader readership.

Below we discuss these changes in further detail, as well as the other
improvements we have done to the paper in response to the editors' and
reviewers' comments. We are pleased with the outcome and hope to have satisfied
all of your concerns. Again, we thank the editor for the opportunity to revise
our manuscript, and the reviewers for their immensely valuable comments.

\vspace{.5cm}

\noindent Sincerely Yours,

\vspace{.5cm}

\noindent The Authors

\newpage

\section*{Editor Comments and Responses}

\textbf{1)} The editor writes: ``I agree with R2 that the framing of the study
could be improved. You currently cite a number of interesting figures and basic
facts about lynching in Brazil in the start of the conclusion. I think this
would fit better in the introduction. R2 also argues that the reference to the
Blair et al study on greater costs of local violence is imprecise. I am
guessing you mean something along the lines of likely to affect a greater
number directly than terrorism or civil war, which I think is reasonable but
could be said more directly.''

\vspace{.3cm}

\noindent \textbf{Response:} 
\begin{quote}

We have moved the facts about lynchings in Brazil to the Introduction and we
agree that the section now provides more context for our experiments. The
second paragraph of the Introduction now reads as follows (page 02):
``\textit{One of the most serious forms of extrajudicial violence is lynching.
  Lynching can be defined as ``incidents of physical violence committed by
  large numbers of private citizens against one or more individuals accused of
  having committed a ``criminal'' offense, whether or not this violence
  resulted in the death of the victim(s)'' \citep[645]{godoy2004justice}.
  Although lynchings occur in more than one hundred countries in all regions of
  the world \citep{jung2020lynching,smith2019contradictions}, Latin America has
  been particularly affected by a sharp increase in vigilante violence.
  Lynching episodes have been reported in Guatemala, Mexico, Peru, Venezuela
  and other countries in the region \citep{barbara2015vigilantes,
  cruz2019determinants, godoy2004justice}. From 2011 to 2015, Brazil registered
  about 2,500 lynching episodes, and 173 people were killed by angry mobs in
  2015 alone---nearly one execution every two days
  \citep{barbara2015vigilantes, oliveira2016mob}. According to José de Souza
  Martins \citeyearpar{martins2015linchamentos}, who has studied lynchings in
  Brazil for more than thirty years, these figures are not only the highest in
  the country's history, but among the highest in the world. The people who
participate in lynchings are typically young men, but they also sometimes
include teenage women and girls, elderly women, and even members of the local
police \citep{moura2017linchamentos}.}

We have also changed the sentence in which we cite \citet{blair2017predicting}
to make our claims more precise. The last sentence in the first paragraph reads
(page 02): ``\textit{This omission is significant not only because local
  violence is pervasive in weak states, but also because it possibly comes at a
greater social cost than terrorism or wars \citep{blair2017predicting}}.''

\end{quote}

\vspace{.3cm}

\noindent \textbf{2)} The editor writes: ``The manuscript is focused primarily
on the empirical approach, but I agree with R1 that you could say a bit more
about the rationale for the covariates considered, perhaps drawing out why more
``novel'' features such as family or type of crime is likely to be important.'' 

\vspace{.3cm}

\noindent \textbf{Response:} 
\begin{quote}

We have followed Reviewer 1's suggestion and have extended section two (``When
Is Lynching Perceived as More Justified?'') to provide further information
about our choice of covariates. The third paragraph of that section now reads
(page 04): ``\textit{In addition to our knowledge of the Brazilian case, we
  also chose our attributes based partly on existing work on crime,
  vigilantism, and extrajudicial violence. From the prison violence literature,
  we know that the relative age differences and the genders of perpetrators and
  victims affect how it is perceived \citep{fleisher2009myth}. We likewise know
  that certain offenses are seen as especially reprehensible
  \citep{skarbek2014social}, so we included such offenses, like molestation,
  alongside less serious offenses. We include race because, in the United
  States, violent lynchings have often been wielded for racist reasons
  \citep{dray2003hands}. Likewise, we include residency because extrajudicial
  violence is often used against people ``who don't belong'' or are
  ``outsiders'' in some respect. Finally, in honor cultures that value
  retaliatory violence, people believe that victims and the family of victims
  have a special right (and often an obligation) to enact retribution
\citep{weiner2013rule}.}

As we show below, we have also included an additional paragraph in the
Conclusion to discuss how our findings relate to the literature on extralegal
violence.

\end{quote}

\vspace{.3cm}

\noindent \textbf{3)} The editor writes: ``On the presentation of the results,
I agree with R2 that the discussion of the results does not need to re-report
all the significance tests, and the full tables could be included in the text
or appendix.''

\vspace{.3cm}

\noindent \textbf{Response:} 
\begin{quote}

We have deleted the statistical details from the main text to improve the
readability of the manuscript. All significance tests are now reported only in
the Supplementary Materials.

\end{quote}

\vspace{.3cm}

\noindent \textbf{4)} The editor writes: ``It would be useful to add a bit more
discussion of Brazil in comparative perspective. I wonder if the results for
Brazil are likely to be more general and that the US could be an outlier, for
example.''

\vspace{.3cm}

\noindent \textbf{Response:} 
\begin{quote}

We have made the changes requested by the editor and Reviewer 1 and our
Conclusion section has been significantly expanded. We are especially grateful
for this suggestion, as it has allowed us to better situate our article in the
broader literate on vigilantism. Paragraph three of the Conclusion, the longest
in the section, is now exclusively dedicated to the scope conditions of our
research. We have linked our findings to important works in the literature and
have also discussed how a culture of honor may explain some of our main
findings. 

The paragraph reads (page 11): ``\textit{These findings have implications that
  extend well beyond Brazil. In particular, our experiments indicate that
  lynchings have a crucial gendered aspect, which has also been documented in
  other contexts. Using data from 18 Latin American countries,
  \citet{nivette2016institutional} also finds that respondents are most likely
  to support lynchings when the criminal raped a child, and studies about
  lynchings in the American South point out that several episodes resulted from
  accusations of sexual assault \citep{jacquet2013giles, smaangs2020race}. We
  believe that a culture of honor may explain these results. Individuals in
  honor societies view crime as an attack on their personal reputation and, in
  turn, are more likely to take revenge to defend their status and that of
  people perceived as deserving protection, such as women and children
  \citep{nisbett2018culture}. A culture of honor may also explain why
  Brazilians see lynchings carried out by the family of the victim as more
  justified, as well as refraining from using extralegal violence if it can
  trigger vendettas. We expect similar results in societies which share those
  cultural norms. Lastly, the fact that race does not appear to be a major
  motivation behind lynchings also reflects the experience of places like Haiti
  or Southern Africa, where popular violence was mainly driven by other social
  factors \citep{berg2011globalizing, jung2020lynching}. Even in the American
  South, Whites and Blacks also lynched people of their own race
  \citep{beck1997race}. In this respect, our paper also highlights that
lynchings in the Global South may be more strongly connected with the idea of
``popular justice'' than with racial animus \citep{martins2015linchamentos}.}''

\end{quote}

\vspace{.3cm}

\noindent \textbf{5)} The editor writes: ``Finally, in order to ensure any
delays on a decision on the resubmission I would encourage you to make sure
that the final manuscript follows the journal guidelines,
\url{https://journals.sagepub.com/author-instructions/rap}. For example, the
sections should not be numbered, and the references should have first initial
only rather than full names and books should have location before publisher
(please ensure that all references are complete – Grimm looks incomplete to me,
for example).''

\noindent \textbf{Response:} 
\begin{quote}

The sections are now unnumbered and we have completed the references according
to Research \& Politics' guidelines.

\end{quote}

\vspace{.3cm}

We would like to thank the editor Kristian Skrede Gleditsch for his very
helpful comments and suggestions.

\section*{Reviewer 1 Comments and Responses}

\noindent \textbf{1)} Reviewer 1 writes: ``The empirical approach is presented
as offering many benefits---I would also acknowledge the drawbacks. And many of
these results do not seem to square with other parts of the world (which is a
feature and I would ask the authors for a little bit of speculation as to why!)
but I would suggest the authors think about the scope conditions imposed by the
setting and how well they seem to extend (the gendered aspect, yes, the race
aspect, perhaps there are some conditions). The finding I think is most
interesting to draw out is the family of rape/murder victims, this is—I think—a
facet not typically focused on and strikes me as the ``newest'' finding and one
I would encourage the authors to pull forward and (in the context of a strict
word limit) emphasize and do a bit of theoretical speculation for others to
build upon!''

\vspace{.3cm}

\noindent \textbf{Response:} 

\begin{quote}
  
Reviewer 1 makes an excellent point. We agree that we had not properly
explained the scope conditions of our research and, as mentioned above, we have
written a new paragraph in the Conclusion to provide further context for our
findings. We compare our results with studies on vigilantism in other Latin
American countries and in the United States, and we also discuss how social
norms, especially those related to a culture of honor, may explain why
Brazilians are more likely to support lynchings when the victim is a woman or a
child. Finally, we also briefly speculate why race does not seem to influence
individual preferences for lynchings in Brazil.

The paragraph reads (page 11): ``\textit{These findings have implications that
  extend well beyond Brazil. In particular, our experiments indicate that
  lynchings have a crucial gendered aspect, which has also been documented in
  other contexts. Using data from 18 Latin American countries,
  \citet{nivette2016institutional} also finds that respondents are most likely
  to support lynchings when the criminal raped a child, and studies about
  lynchings in the American South point out that several episodes resulted from
  accusations of sexual assault \citep{jacquet2013giles, smaangs2020race}. We
  believe that a culture of honor may explain these results. Individuals in
  honor societies view crime as an attack on their personal reputation and, in
  turn, are more likely to take revenge to defend their status and that of
  people perceived as deserving protection, such as women and children
  \citep{nisbett2018culture}. A culture of honor may also explain why
  Brazilians see lynchings carried out by the family of the victim as more
  justified, as well as refraining from using extralegal violence if it can
  trigger vendettas. We expect similar results in societies which share those
  cultural norms. Lastly, the fact that race does not appear to be a major
  motivation behind lynchings also reflects the experience of places like Haiti
  or Southern Africa, where popular violence was mainly driven by other social
  factors \citep{berg2011globalizing, jung2020lynching}. Even in the American
  South, Whites and Blacks also lynched people of their own race
  \citep{beck1997race}. In this respect, our paper also highlights that
lynchings in the Global South may be more strongly connected with the idea of
``popular justice'' than with racial animus \citep{martins2015linchamentos}.}''

We have added three sentences on pages 6 and 7 to offer readers more context
about our covariates. They read (pages 6--7): ``\textit{Taken together, the
  results provide experimental evidence that support for lynchings in Brazil
  does not resemble the typical racial patterns scholars have observed in the
  United States \citep{dray2003hands,obert2018keeping,seguin2019national}.
  These findings are consistent with recent research on vigilantism in Haiti
  \citep{jung2020lynching} and with journalistic observations. They further
  suggest that the US experience with lynchings might be distinct from
vigilantism in other places and times \citep{oliveira2016mob}.}'' 

Finally, we have added two sentences on page 8 to discuss how cultural norms
are relevant to our results. They read (page 8): ``\textit{In sum,
respondents did not believe that lynchings should be carried out by the state
but did believe they should be used as a tool for individual or family
retribution. These results are consistent with norms of an honor culture in
which offenses are seen to tarnish the victim's status and the only way to
remove the stigma is through self-help efforts in the form of retaliatory
violence \citep{nisbett2018culture}.}''

\end{quote}

\vspace{.3cm}

\noindent \textbf{2)} Reviewer 1 writes: ``With respect to experiment two---I
would be very clear about the control group (in my reading) having been
``treated'' by the conjoint. In terms of estimation this is fine, but in terms
of baseline attitudes that we might use to benchmark in future studies, I do
worry a bit.  I would also encourage a bit more description of the subgroup
analyses on experiment 2 since I think they are likely interesting/important
for future work.''

\vspace{.3cm}

\noindent \textbf{Response:} 
\begin{quote}

We agree that our explanations regarding our experimental design were
incomplete. We randomized the order of the experiments precisely to avoid
carryover effects, thus we hope our estimates are unbiased because of this
procedure. We have added a footnote to the Supplementary Material to explain
this point. It reads (page 83 of the SM, footnote number 02): ``\textit{To
  prevent eventual carryover effects caused by the conjoint, we randomized the
  order of the conjoint and the information provision experiments
\citep{perreault1975controlling}.}''.

We also agree that the subgroup analyses may be interesting to scholars. Our
Supplementary Materials now include a short comment about each of our subgroup
tests. We note when the results are statistically significant and when they are
vary according to the covariates we control for. For instance, Section C.4.1 of
the Supplementary Materials, in which we assess whether our results for the
conjoint experiment differs between genders, we write the following (page 24):
``\textit{Results do not seem to vary according to the gender of the
  respondent. We focus here on the differences between males and females and
  exclude the 11 observations in which respondents preferred not to say their
  gender or marked ``other'' in our questionnaire. Across all conjoint
experiment attributes, we see an overlap between the 95\% confidence intervals
for males and females.}'' Similarly, in Section C.4.3, in which we test whether
our results vary by race, we write: ``\textit{Below are our results when we
disaggregate the data by race. We find that they are almost identical is all
dimensions except for offense. Asian respondents are much less likely to select
profiles that contain pickpocketing as a crime.}''  All analyses are also
accompanied by their corresponding \texttt{R} code to facilitate the
replication of our results.

\end{quote}

\vspace{.3cm}

\noindent \textbf{3)} Reviewer 1 writes: ``Minor: 1) I recommend citing Nick
Rush Smith and there is a new special issue out at Comparative Politics that
looks to have several pieces that are relevant! 2) I would like to see the
supplementary materials included!''

\vspace{.3cm}

\noindent \textbf{Response:} 
\begin{quote}

We have added a citation to Nick Rush Smith's work on page 02 and a citation to
the special issue in footnote 01 (page 02). The Supplementary
Materials are now included in the submission as well.

\end{quote}

\vspace{.3cm}

We would like to thank Reviewer 1 for her/his helpful comments and suggestions.

\section*{Reviewer 2 Comments and Responses}

\textbf{Major issues:}

\noindent \textbf{1)} Reviewer 2 writes: ``Framing. The abstract is rather
short, and I believe R&P abstracts can be up to 200 words. It would be good to
make use of the space to highlight additional results that are especially
counterintuitive or interesting.''

\vspace{.3cm}

\noindent \textbf{Response:} 
\begin{quote}

We agree that the abstract could be improved. We have added two sentences to
the revised abstract and it now reads as follows: ``\textit{Why do people
support extrajudicial violence? In two survey experiments with respondents in
Brazil, we examine which characteristics of lynching scenarios garner greater
support for lynching and whether providing different types of information about
lynching reduces support for it. We find that people often do support community
members to take vengeance. In particular, our analysis finds that people
strongly support the use of extrajudicial violence by families of victims
against men who sexually assault and murder women and children. We also find
that criminal punishment and the threat of vendettas reduce support, but
appeals to the human rights of victims have zero effect on support for
lynchings. Unlike the U.S. experience with lynchings, race was not observed to
play an important role in how respondents answered the survey.}''

\end{quote}

\vspace{.3cm}

\noindent \textbf{2)} Reviewer 2 writes: ``The introduction motivates the study
by noting that ``local violence'' can be more costly than war or terrorism
violence. This raises questions about what is ``local violence'', and I'm not
sure the cited source after this statement clarifies the situation. The cited
Blair at al piece indicates that local violence is ``possibly'' more costly
than war or terrorism, but this seems imprecise to me. If there is more space
to defend this statement more, that might be ok, but it is probably better just
to cut it and motivate the study in other ways.''

\vspace{.3cm}

\noindent \textbf{Response:} 
\begin{quote}

  

\end{quote}

\vspace{.3cm}

\noindent \textbf{3)} Reviewer 2 writes: ``Why is the scale of disagree to
agree 0-100? Could the authors cite a source or sources on why this precise
range was used, and briefly discuss some tradeoffs with such a scale? Why not
a, for example, seven-point Likert scale?''

\vspace{.3cm}

\noindent \textbf{Response:} 
\begin{quote}

\end{quote}

\vspace{.3cm}

\noindent \textbf{4)} Reviewer 2 writes: ``Related to the above, the wording of
the instructions seem to suggest that respondents could select anywhere between
0 and 49 if they disagree; unless I am mistaken, the instructions do not
indicate, for example, that 0 indicates fully disagree and 100 indicates fully
agree. Were any more detailed instructions provided? Is there any evidence that
people fully understood the instructions in this regard?''

\vspace{.3cm}

\noindent \textbf{Response:} 
\begin{quote}

\end{quote}

\vspace{.3cm}

\noindent \textbf{5)} Reviewer 2 writes: ``In the results section, it is
probably not necessary to list the coefficient, SE, and p-value in the text for
so many variables when the figures show the most important information,
assuming the full tables are in the supplementary information.'' 

\vspace{.3cm}

\noindent \textbf{Response:} 
\begin{quote}

\end{quote}

\vspace{.3cm}

\noindent \textbf{6)} Reviewer 2 writes: ``There is other recent survey work on
support for vigilantism that could be acknowledged, like Zizumbo-Colunga 2017.
On Brazil, there is Schuberth and others. Beyond survey work, there is a
growing line of work on vigilantes or autodefensas in other Latin American
countries, such as Peru or Mexico, that could be briefly discussed. For example
Kloppe-Santamaria on Puebla in JLAS or Trevizo on Mexico generally in LAPS.
This work seems relevant in many ways.''

\vspace{.3cm}

\noindent \textbf{Response:} 
\begin{quote}
    Following Reviewer 2's recommendations, 

\end{quote}

\vspace{.3cm}

\noindent \textbf{7)} Reviewer 2 writes: ``Very minor note: on page 12,
significantly effect should be significantly affect.''

\vspace{.3cm}

\noindent \textbf{Response:} 
\begin{quote}
    We have corrected this typo in the revised version of the paper.
\end{quote}

\vspace{.3cm}

We would like to thank Reviewer 2 for her/his helpful comments and suggestions.

\newpage
\bibliography{../article/references.bib}
\bibliographystyle{apalike}

\end{document}d
