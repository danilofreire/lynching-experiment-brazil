\documentclass[a4paper,12pt]{article}
\usepackage{anysize}
\usepackage[T1]{fontenc}
\usepackage[stable]{footmisc}
\usepackage{setspace}
\usepackage{lmodern}
\usepackage{libertine}
\usepackage[libertine]{newtxmath}
\usepackage[scale=0.825]{FiraMono}
\usepackage[top=2cm,bottom=2cm,left=2cm,right=2cm]{geometry}
\usepackage{mathtools}
\usepackage[authoryear]{natbib}
\usepackage[UKenglish]{babel}
\usepackage[UKenglish]{isodate}
\usepackage{babelbib}
\usepackage{graphicx}
\usepackage{booktabs, makecell, longtable}
\usepackage{dcolumn}
\usepackage{float}
\usepackage[caption = false]{subfig}
\floatplacement{figure}{H}
\usepackage{caption}
\usepackage{rotating}
\usepackage{pdflscape}
\usepackage{pdflscape}
\usepackage{ifthen}
\usepackage{graphicx}
\usepackage[usenames,dvipsnames]{xcolor}
\definecolor{darkblue}{rgb}{0.0,0.0,0.55}
\usepackage{tikz}
\usetikzlibrary{shapes.geometric, arrows}
\setcitestyle{aysep={}}
\usepackage{etoolbox}
\makeatletter
\patchcmd{\NAT@citex}
  {\@citea\NAT@hyper@{%
	 \NAT@nmfmt{\NAT@nm}%
	 \hyper@natlinkbreak{\NAT@aysep\NAT@spacechar}{\@citeb\@extra@b@citeb}%
	 \NAT@date}}
  {\@citea\NAT@nmfmt{\NAT@nm}%
   \NAT@aysep\NAT@spacechar\NAT@hyper@{\NAT@date}}{}{}
\patchcmd{\NAT@citex}
  {\@citea\NAT@hyper@{%
	 \NAT@nmfmt{\NAT@nm}%
	 \hyper@natlinkbreak{\NAT@spacechar\NAT@@open\if*#1*\else#1\NAT@spacechar\fi}%
   {\@citeb\@extra@b@citeb}%
	 \NAT@date}}
  {\@citea\NAT@nmfmt{\NAT@nm}%
   \NAT@spacechar\NAT@@open\if*#1*\else#1\NAT@spacechar\fi\NAT@hyper@{\NAT@date}}
  {}{}
\makeatother
\cleanlookdateon
\exhyphenpenalty=1000
\hyphenpenalty=1000
\widowpenalty=1000
\clubpenalty=1000
\usepackage{hyperref}

\hypersetup{
	breaklinks=true,
	linkcolor=Mahogany,
	citecolor=Mahogany,
	urlcolor=darkblue,
	colorlinks=true}

\newcommand*\justify{%
  \fontdimen2\font=0.4em% interword space
  \fontdimen3\font=0.2em% interword stretch
  \fontdimen4\font=0.1em% interword shrink
  \fontdimen7\font=0.1em% extra space
  \hyphenchar\font=`\-% allowing hyphenation
}

\begin{document}

\doublespacing

\noindent \textbf{Research \& Politics}

\noindent \today 

\vspace{.5cm}

\noindent Dear Editor and Reviewers,

\vspace{.5cm}

We would like to thank you for the opportunity to revise our manuscript,
``Vigilantism and Institutions: Understanding Attitudes toward Lynching in
Brazil'' (Ms. No. RAP-22-0164). Following the comments and suggestions of the
editors and reviewers, we have thoroughly revised our article and we believe it
has improved significantly as a result. 

We were especially committed to discussing the scope conditions of our article
and to clarifying some technical questions about our experiments as highlighted
by Reviewers 1 and 2. We did this by editing the Introduction, adding a new
paragraph to the Conclusion, and including more details about our methodology
in section two and in the Supplementary Materials. We also removed the
significance tests from the manuscript, as per the suggestion of Reviewer 2,
and added a footnote in the Supplementary Materials indicating that we
randomized the order of the experiments to reduce the risk of order effects.
Moreover, we now show how the findings we report in the article compare with
those found in other contexts, as suggested by Reviewer 1. We also discuss the
results of our subgroup analyses in the Supplementary Materials, again
following the recommendation of Reviewer 1. Lastly, we include all references
suggested by Reviewer 1 and 2 in the main body of the manuscript. We are
extremely appreciative of these recommendations and agree that the manuscript
is now stronger and appeals to a broader readership.

Below we discuss these changes in further detail, as well as the other
improvements we have done to the paper in response to the editors' and
reviewers' comments. We are pleased with the outcome and hope to have satisfied
all of your concerns. Again, we thank the editor for the opportunity to revise
our manuscript, and the reviewers for their immensely valuable comments.

\vspace{.5cm}

\noindent Sincerely yours,

\vspace{.5cm}

\noindent The Authors

\newpage

\section*{Editor Comments and Responses}

\textbf{1)} The editor writes: ``I agree with R2 that the framing of the study
could be improved. You currently cite a number of interesting figures and basic
facts about lynching in Brazil in the start of the conclusion. I think this
would fit better in the introduction. R2 also argues that the reference to the
Blair et al study on greater costs of local violence is imprecise. I am
guessing you mean something along the lines of likely to affect a greater
number directly than terrorism or civil war, which I think is reasonable but
could be said more directly.''

\vspace{.3cm}

\noindent \textbf{Response:} 
\begin{quote}

We have moved the facts about lynchings in Brazil to the Introduction and we
agree that the section now provides more context for our experiments. The
second paragraph of the Introduction is as follows: ``\textit{One of the most
serious forms of extrajudicial violence is lynching. Lynching can be defined
as ``incidents of physical violence committed by large numbers of private
citizens against one or more individuals accused of having committed a
``criminal'' offense, whether or not this violence resulted in the death of
the victim(s)'' \citep[645]{godoy2004justice}. Although lynchings occur in
more than one hundred countries in all regions of the world
\citep{jung2020lynching,smith2019contradictions}, Latin America has been
particularly affected by a sharp increase in vigilante violence. Lynching
episodes have been reported in Guatemala, Mexico, Peru, Venezuela and other
countries in the region \citep{barbara2015vigilantes, cruz2019determinants,
godoy2004justice}. From 2011 to 2015, Brazil registered about 2,500 lynching
episodes, and 173 people were killed by angry mobs in 2015 alone---nearly one
execution every two days \citep{barbara2015vigilantes, oliveira2016mob}.
According to José de Souza Martins \citeyearpar{martins2015linchamentos}, who
has studied lynchings in Brazil for more than thirty years, these figures are
not only the highest in the country's history, but among the highest in the
world. The people who participate in lynchings are typically young men, but
they also sometimes include teenage women and girls, elderly women, and even
members of the local police \citep{moura2017linchamentos}.}

We have also changed the sentence in which we cite \citet{blair2017predicting}
to make our claims more explicit. The last sentence in the first paragraph now
reads: ``\textit{This omission is significant not only because local violence
is pervasive in weak states, but also because it often comes at a greater
social cost than terrorism or civil wars \citep{blair2017predicting}}.''

\end{quote}

\vspace{.3cm}

\noindent \textbf{2)} The editor writes: ``The manuscript is focused primarily
on the empirical approach, but I agree with R1 that you could say a bit more
about the rationale for the covariates considered, perhaps drawing out why more
``novel'' features such as family or type of crime is likely to be important.'' 

\vspace{.3cm}

\noindent \textbf{Response:} 
\begin{quote}

We have followed Reviewer 1's suggestion and have extended section two (``When
Is Lynching Perceived as More Justified?'')  to provide further information
about our choice of covariates. The third paragraph of that section now reads:
``\textit{In addition to our knowledge of the Brazilian case, we also chose our
attributes based partly on existing work on crime, vigilantism, and
extrajudicial violence. From the prison violence literature, we know that the
relative age differences and the genders of perpetrators and victims affect
how it is perceived \citep{fleisher2009myth}. We likewise know that certain
offenses are seen as especially reprehensible \citep{skarbek2014social}, so
we included such offenses, like molestation, alongside less serious offenses.
We include race because, in the United States, violent lynchings have often
been wielded for racist reasons \citep{dray2003hands}. Likewise, we include
residency because extrajudicial violence is often used against people ``who
don't belong'' or are ``outsiders'' in some respect. Finally, in honor
cultures that value retaliatory violence, people believe that victims and the
family of victims have a special right (and often an obligation) to enact
retribution \citep{weiner2013rule}.}

As we show below, we have also included an additional paragraph in the
Conclusion to discuss how our findings relate to the literature on extralegal
violence.

\end{quote}

\vspace{.3cm}

\noindent \textbf{3)} The editor writes: ``On the presentation of the results,
I agree with R2 that the discussion of the results does not need to re-report
all the significance tests, and the full tables could be included in the text
or appendix.''

\vspace{.3cm}

\noindent \textbf{Response:} 
\begin{quote}

We have deleted the statistical details from the main text to improve the
readability of the manuscript. All significance tests are now reported only in
the Supplementary Materials.

\end{quote}

\vspace{.3cm}

\noindent \textbf{4)} The editor writes: ``It would be useful to add a bit more
discussion of Brazil in comparative perspective. I wonder if the results for
Brazil are likely to be more general and that the US could be an outlier, for
example.''

\vspace{.3cm}

\noindent \textbf{Response:} 
\begin{quote}

We have made the changes requested by the editor and Reviewer 1 and our
Conclusion section has been significantly expanded. We are especially grateful
for this suggestion, as it has allowed us to better situate our article in the
broader literate on vigilantism. Paragraph three of the Conclusion, the longest
in the section, is now exclusively dedicated to the scope conditions of our
research. We have linked our findings to important works in the literature and
have also discussed how a culture of honor may explain some of our main
findings. 

The paragraph reads: ``\textit{These findings have implications that extend
well beyond Brazil. In particular, our experiments indicate that lynchings
have a crucial gendered aspect, which has also been documented in other
contexts. Using data from 18 Latin American countries,
\citet{nivette2016institutional} also finds that respondents are most likely
to support lynchings when the criminal raped a child, and studies about
lynchings in the American South point out that several episodes resulted from
accusations of sexual assault \citep{jacquet2013giles, smaangs2020race}. We
believe that a culture of honor may explain these results. Individuals in
honor societies view crime as an attack on their personal reputation and, in
turn, are more likely to take revenge to defend their status and that of
people perceived as deserving protection, such as women and children
\citep{nisbett2018culture}. A culture of honor may also explain why
Brazilians see lynchings carried out by the family of the victim as more
justified, as well as refraining from using extralegal violence if it can
trigger vendettas. We expect similar results in societies which share those
cultural norms. Lastly, the fact that race does not appear to be a major
motivation behind lynchings also reflects the experience of places like Haiti
or Southern Africa, where popular violence was mainly driven by other social
factors \citep{berg2011globalizing, jung2020lynching}. Even in the American
South, Whites and Blacks also lynched people of their own race
\citep{beck1997race}. In this respect, our paper also highlights that
lynchings in the Global South may be more strongly connected with the idea of
``popular justice'' than with racial animus \citep{martins2015linchamentos}.}''

\end{quote}

\vspace{.3cm}

\noindent \textbf{5)} The editor writes: ``Finally, in order to ensure any
delays on a decision on the resubmission I would encourage you to make sure
that the final manuscript follows the journal guidelines,
\url{https://journals.sagepub.com/author-instructions/rap}. For example, the
sections should not be numbered, and the references should have first initial
only rather than full names and books should have location before publisher
(please ensure that all references are complete – Grimm looks incomplete to me,
for example).''

\noindent \textbf{Response:} 
\begin{quote}

We have completed the references according to Research \& Politics' guidelines
and the sections are unnumbered as requested.

\end{quote}

\vspace{.3cm}

We would like to thank the editor Kristian Skrede Gleditsch for his very
helpful comments and suggestions.

\section*{Reviewer 1 Comments and Responses}

\noindent \textbf{1)} Reviewer 1 writes: ``The empirical approach is presented
as offering many benefits---I would also acknowledge the drawbacks. And many of
these results do not seem to square with other parts of the world (which is a
feature and I would ask the authors for a little bit of speculation as to why!)
but I would suggest the authors think about the scope conditions imposed by the
setting and how well they seem to extend (the gendered aspect, yes, the race
aspect, perhaps there are some conditions).''

\vspace{.3cm}

\noindent \textbf{Response:} \begin{quote}
  
Reviewer 1 makes an excellent point. We agree that we had not properly
explained the scope conditions of our research and, as mentioned above, we have
included a long paragraph in the Conclusion to detail 

\end{quote}

\vspace{.3cm}

\noindent \textbf{2)} Reviewer 1 writes: ``In the empirical exercise, the authors routinely reference separating the empirical studies they use as their sample based on the "main explanatory variable." For instance, in section 3.3. This strikes me as problematic (or at least poorly explained) because it is typically standard in this literature (at least for those studies on bicameral legislatures such as cross-state studies on the US) to include both the upper and lower chamber size variables. So it is unclear to me how such a study would have a "main explanatory variable." This needs to be more clearly explained.''

\vspace{.3cm}

\noindent \textbf{Response:} 
\begin{quote}
    We agree that our explanations regarding the choice and use of the variables of interest were insufficient. When studies use evidence from bicameral legislatures, we included the coefficients for both upper and lower chamber size in all our samples. This is the case for thirteen articles in our analysis. Using data from both chambers is one of the reasons why our restricted sample contains 45 coefficients for a total of 30 papers. We added clarifications regarding our choice of independent variable in the first paragraph of the Data and Methods section, on page four.
    
    A specific comment on variable choice in bicameral systems is in footnote \#2. It reads: ``\textit{Since much of the literature estimates how institutional designs affect this relationship, ours and many other articles use both lower and upper chamber sizes as main explanatory variables.}''
\end{quote}

\vspace{.3cm}

\noindent \textbf{3)} Reviewer 1 writes: ``In Table 3, the authors also only present N and log(N) results; K is not indicated—was it an excluded group? If so, this needs to be made clear.''

\vspace{.3cm}

\noindent \textbf{Response:} 
\begin{quote}
    We acknowledge that Table 3 did not display the reference categories clearly enough. We have highlighted the notes below the table so the reader can have a more straightforward interpretation of the estimates. They present the reference category for each variable and moderator in the meta-regressions. For the independent variable, the reference category was upper chamber size, which is why this category was omitted in the table.
\end{quote}

\vspace{.3cm}

\noindent \textbf{4)} Reviewer 1 writes: ``Further, it is not clear to the reader why the authors use lower chamber size, upper chamber size, and then a log of only lower chamber size. Are there no studies which employ a log of upper chamber size?''

\vspace{.3cm}

\noindent \textbf{Response:} 
\begin{quote}
    Following the reviewer's remark, we double checked our study sample and confirm that the natural logarithm of upper chamber size is not present in any papers. As mentioned above, we included clarifications about our choice of independent variable in the first paragraph of the Data and Methods section (p.4). Footnote \#2 reads: ``\textit{We did not find any article that used the natural logarithm of upper chamber size in their models.}''
\end{quote}

\vspace{.3cm}

\noindent \textbf{5)} Reviewer 1 writes: ``How did the authors handle those studies in this literature which focused on unicameral legislatures? Were these grouped in with the "lower chamber" studies? Were they excluded? This is not clearly explained in the paper. In general, the results on unicameral legislatures tend to support the law of $1/n$.''

\vspace{.3cm}

\noindent \textbf{Response:} 
\begin{quote}
    As noted above, we agree that important details about data collection and coding were missing. Unicameral legislature sizes were coded under lower chamber size or natural logarithm of lower chamber size. Many studies consider lower chambers and unicameral legislatures as analogous institutions for quantitative purposes. An example is one of the papers in our meta-analysis. We expand the explanations surrounding the treatment of independent variables in the first paragraph of the Data and Methods section. A specific clarification on unicameral legislatures is in Footnote \#2. It reads: ``\textit{There are a few important nuances concerning coding of these variables. Unicameralism, for example, is captured both by lower chamber size ($n = 7$) and by log lower chamber size ($n = 5$).}''
    
    Upon reading this comment, we also acknowledged that although the differences between upper and lower chambers were so relevant to the discussion, we had not given enough attention to unicameral legislatures. Inspired by this remark, we coded a new moderator for institutional design. The categories were `Unicameral', for coefficients from purely unicameral systems; `Bicameral', for coefficients from purely bicameral systems; and `Mixed', for coefficients from studies that used samples which contained a mix of unicameral and bicameral systems. Examples of the latter were all the papers that used country-level data. Our findings confirm the Reviewer's insight: in the meta-regressions that aggregate all coefficients, unicameralism is a significant positive predictor of the $1/n$ effect. 
    
    Since this is an important finding, we discuss the Institutional Design moderator in all sections. The Data and Methods section presents under which category each article fits in Table 1, and its descriptive statistics in Table 2. In the first paragraph of the Meta-Regressions subsection, we explain how the variable is coded (p.12), and briefly interpret its results in the last paragraph (p.14).
    
    Because we found such interesting results in the meta-regressions, we decided to perform subgroup analyses with heterogeneous effects for unicameralism as well. The results can be found in section H.13 in the Supplementary Materials. We undertook tests similar to those we present in the article for Estimation Methods (p.12). Using the lower chamber size $\times$ per capita expenditure combination, we aggregated the data in two groups: one just for unicameral systems, and the other for all institutional designs excluding unicameralism. For unicameral systems, the estimates were mostly positive in the restricted sample, but none was statistically significant. Results were overwhelmingly positive for the extended sample, where they almost reached statistical significance. These results are in line with what we found in the rest of the paper, however they do not help us reach more robust conclusions on the effect of institutional design on the ``law of $1/n$''.
\end{quote}

\vspace{.3cm}

\noindent \textbf{6)} Reviewer 1 writes: ``The Supplementary Materials is full of content that is of very little use to the reader; even as an online appendix, there is no need to include roughly 100 pages of code from a statistical software. I believe the most relevant Supplementary Materials is the detailed discussion on selection.''

\vspace{.3cm}

\noindent \textbf{Response:} 
\begin{quote}
    We agree that the R script is useful to a limited amount of scholars, and not to the vast majority of readers. As mentioned above, we have separated the Supplementary Materials in two parts. The first follows Reviewer 1's recommendation, including a discussion of the theory, the study selection procedures, a description of how we coded the data in detail, and analysis results. The second gathers all scripts we use in each step of the study, and is aimed at those wishing to replicate our work or to produce new meta-analyses in the social sciences.
\end{quote}

\vspace{.3cm}

We would like to thank Reviewer 1 for her/his helpful comments and suggestions.

\section*{Reviewer 2 Comments and Responses}

\textbf{Major issues:}

\noindent \textbf{1)} Reviewer 2 writes: ``I believe the main framing of this manuscript should focus on explaining the interesting variation between studies (i.e., when do the law of 1/n or the reverse law are expected). From this perspective, the introduction part of the paper currently quickly jumps to the technicalities of the meta-analysis (e.g., a description of the two models used, measurement of DV etc.; see p. 2). These technical descriptions should appear in the method section. Instead, it would be more helpful to see some theoretical discussion on the reasons the literature proposes for the heterogeneity found in this study - even in a concise manner.''

\vspace{.3cm}

\noindent \textbf{Response:} 
\begin{quote}
    We agree that the technical discussion in the introductory section was misplaced. We have restructured the Introduction to fit Reviewer 1 and Reviewer 2's recommendations. As displayed above (p.4 of this letter), we expanded the literature review to capture the main points in the debate that dialogue with our findings. We have also made our critiques of the scholarship more clear, explicitly pointing to which we thought were the sources of heterogeneity in previous work in the Discussion section. 
\end{quote}

\vspace{.3cm}

\noindent \textbf{2.1)} Reviewer 2 writes: ``It is currently difficult to understand why many studies included in the meta-analysis have several effect sizes. Is it because of using different dependent? Some examples would help the readers.''

\vspace{.3cm}

\noindent \textbf{Response:} 
\begin{quote}
    In our restricted sample, a study can supply more than one coefficient if it employs more than one dependent or independent variable. Studies that analyse bicameral systems use upper chamber size and lower chamber size as independent variables, and thus supply two coefficients each. 
\end{quote}

\vspace{.3cm}

\noindent \textbf{2.2)} Reviewer 2 writes: ``Moreover, and more substantially, I am not sure how including studies with several effect sizes in a separate model (p. 7) can solve the violation of non-independent effect sizes, required in meta-analyses. The authors do not report the statistical models they used for solving these violations (I did not find it in the appendix either). There are various ways to address such issues, such as robust variance estimation (see Hedges et al., 2010) or using multi-level meta-analyses (see e.g. application in Matthes et al., 2019). 

Furthermore, since there were several studies focusing on the same country (e.g., 14 studies investigating the US), did they use the same samples in their analysis? If so, doesn't it violate the non-independence assumption on its own (as they use the same district ``sample'')? If that is the case, the authors should treat this effect size dependency also for the first model (with the 42 main coefficients). These issues are very important since some of the authors' findings are based only on the ``extended models'' in Table 3 (i.e., those including 142 effect sizes, even from studies reporting several effect sizes related to the same sample). If these findings are based on biased estimates, we should be a lot more cautious interpreting them.''

\vspace{.3cm}

\noindent \textbf{Response:} 
\begin{quote}
    As Reviewer 2 correctly notes, the coefficient extraction procedure we just presented in the previous response is the main source of effect size dependence in our extended sample analyses. An additional source of dependence is the fact that many studies analyse the same cases, sometimes even using the same data sources. The Reviewer is also correct to point out that separating the effect sizes into a restricted and an extended sample does not properly address the dependence of the coefficients within the extended sample. We followed the Reviewer's suggestion rigorously by employing multilevel random effect models in all of our estimations. A thorough explanation of how we identified, grouped the studies that constitute the new levels, and coded them can be found in section H.1 of the Supplementary Materials. We describe the procedure more succinctly in the Data and Methods section in the main paper (last paragraph on page 7):
    
    ``\textit{We add two extra levels to the regular meta-analysis, one including a unique publication ID for each paper, and another indicating the data source used in the original study. By adding these two levels, we account for within- and between-study variation, thus removing these sources of effect size dependency and improving the accuracy of the results.}''
    
    We believe the inclusion of additional levels reduces the bias previously present in our analyses, producing far more reliable estimates. We are highly appreciative of this comment, as it has ignited major changes in our manuscript, severely improving the robustness of our results. 
\end{quote}

\vspace{.3cm}

\noindent \textbf{3)} Reviewer 2 writes: ``How did the authors convert the original studies' coefficients into standardized effect sizes? It is important to mention how the authors render these coefficients comparable due to the various research designs included in the meta-analysis.''

\vspace{.3cm}

\noindent \textbf{Response:} 
\begin{quote}
    We have included a paragraph in the Data and Methods section (p.7) to clarify the coefficient standardisation procedure. It reads:
    
    ``\textit{We use Hedges' $g$ to calculate effect }''
\end{quote}

\vspace{.3cm}

\noindent \textbf{4)} Reviewer 2 writes: ``our results suggest that study coefficients are highly sensitive to research design choices'' (p. 13) - this is a rather bold statement (``highly sensitive'') since most differences between study designs are nonsignificant (on Table 3). I would tone down this argument much more and address its limitations in the discussion.'' 

\vspace{.3cm}

\noindent \textbf{Response:} 
\begin{quote}
    We have adapted the language when referring to the effects of estimation methods according to the Reviewer's suggestions throughout the paper. For instance, paragraph 6 of the Introduction (p.3) reads:
    
    ``\textit{However, when we look only at articles that employ regression discontinuity designs, we see that all four papers included in our sample suggest that a hi}''
    
    The Results and Discussion sections present these findings more objectively. On page 14, we report the following conclusion regarding estimation methods: ``\textit{The meta-regressions confirm that modern estimation methods such as RDDs and panel/fixed-effects models decrease effects more frequently than OLS regressions.}''
\end{quote}

\vspace{.3cm}

\noindent \textbf{5)} Reviewer 2 writes: ``I encourage the authors to elaborate more on the implications of their results rather than to simply describe them, especially in the context of their moderators (electoral systems and possibly, the research design - but see my above concerns regarding the results related to study designs). Although the format should be concise, such a discussion will improve the readers' understanding of the main findings much more (especially non-expert readers). For example, the seemingly "methodological" explanation of research design as a moderator has substantial implications. If that is the case, to what extent can we lean on earlier studies that did not use causal inference?''

\vspace{.3cm}

\noindent \textbf{Response:} 
\begin{quote}
    Following Reviewer 2's recommendations, we have more appropriately framed the paper so that we discuss the institutional moderators and their relationship to the theory across the entire piece. Since we changed our model specifications and added a new moderator, our Results section was also significantly altered. We reduced the Meta-Analysis subsection (3.2) and expanded the Meta-Regressions subsection (3.3) to account for the most relevant and auspicious findings. In this sense, in paragraph 2 of subsection 3.3, we explain to the reader how she should interpret the meta-regression results. We believe this makes our arguments about the meta-regression results more compelling. We also added more substantive insights to accompany the description of meta-regression results in each paragraph of section 3.3. 
    
    We present and partly interpret our results in the Introduction section, reporting back to elements from the literature and to the theory. The last paragraph in the Introduction (p.3) reads:
    
    ``\textit{The meta-regressions provide additional evidence that our study sample is highly heterogeneous and that effect sizes differ substantially according to study specifications. When using an extended sample of 162 coefficients, we find that unicameralism is associated with higher public spending, as predicted by the ``law of $1/n$''. Since most unicameral legislatures in our sample are local governments (municipalities or districts), this result supports the predictions of Moreover, our meta-regressions indicate that larger upper chambers spend more in terms of per capita expenditure than lower chambers, a result that also appears in the binomial tests. Overall, non-majoritarian voting systems seem to decrease government spending, following the idea that the $1/n$ effect grows weaker as the empirical cases distance from the original definition of the law. Finally, meta-regression results confirm that regression discontinuity designs reduce public spending estimates.}''
    
    The Discussion section has also increased both in size and scope. Now, we more aptly establish the dialogue between our electoral system findings and the original theory and literature. For instance, in page 15, the following reads:
    
    ``\textit{While we believe that moving beyond the majoritarian districts framework could produce valuable insights, institutional features that are central to the theory cannot be overrun. For example, proportional representation (PR) electoral systems allow candidates whose constituents are spread across large territories to provide diffused public goods and win elections. However, geographically-targeted service provision is at the very core of the legislative behaviour that produces the "law of $1/n$". Thus, scholars should consider the possible implications of these micro-level dynamics when applying the "law of $1/n$" logic to different settings.}''
\end{quote}

\vspace{.3cm}

\noindent \textbf{More minor comments:}

\vspace{.3cm}

\noindent \textbf{1)} Reviewer 2 writes: ``Table 1 - it will be clearer to separate the notes at the bottom of the table into separate columns. For instance, note \#1 will explain the abbreviations of the Journal column note \#2 will refer to the Country column. Otherwise, the reader has to manually look for a given abbreviation out of the rather long list.''

\vspace{.3cm}

\noindent \textbf{Response:} 
\begin{quote}
    We agree this can be an unnecessary inconvenience to the reader. We have changed the notes in Table 1 according to the Reviewer's specifications. 
\end{quote}

\vspace{.3cm}

\noindent \textbf{2)} Reviewer 2 writes: ``Also regarding Table 1, I believe it will be more convenient to simply write the authors and the year of publication in the first column (e.g., "Stein et al., 1998") while keeping the table order as it is (according to the year of publication) - rather than splitting them into two columns. I am also not sure whether the title of the paper really helps the readers or simply confuses them (it makes the table less elegant in my opinion). I would delete the title (if a reader wishes to see the full reference, she can look it in the bibliography). In contrast, it will be more beneficial to mention inside the table the DVs used in each study (as currently explained in the body text; p. 6).''

\vspace{.3cm}

\noindent \textbf{Response:} 
\begin{quote}
    We have made these alterations to Table 1 accordingly. We also added a column for the new Institutional Design moderator.
\end{quote}

\vspace{.3cm}

\noindent \textbf{3)} Reviewer 2 writes: ``The use of $n$ and $k$ as signifiers of the independent variables is a bit confusing throughout the paper since in the meta-analyses domain, they usually signify the number of effect sizes ($k$) and the number of respondents per study ($n$). More broadly, it is even more confusing when discussing Table 3 (the reader has to remember what $n$ and $k$ represent, instead of simply writing the meanings) or when interpreting its results in the discussion part.''

\vspace{.3cm}

\noindent \textbf{Response:} 
\begin{quote}
    We agree that adding labels to variables is an unnecessary inconvenience to the reader. All references to the independent variables now translate their full meaning.
\end{quote}

\vspace{.3cm}

\noindent \textbf{4)} Reviewer 2 writes: ``I appreciate the authors' elaborated appendix, but the readers might get a bit lost without specific references to the sub-sections in it throughout the main paper: for example, not just referring to the Supplementary Materials in general, but to the specific section in it. Moreover, the authors should consider leaving out the R code from the online appendix, and simply anonymously upload it to an external open science source. For people who are not experts in R (and perhaps also for those who are), the combination of code and text can be a bit confusing.''

\vspace{.3cm}

\noindent \textbf{Response:} 
\begin{quote}
    As noted above, we agree that only a handful of scholars would be interested in R scripts, which get in the way of most readers. We have split the Supplementary Materials into two documents. The one we refer to as the ``Supplementary Materials'' includes a discussion of the theory, the study selection procedures, a description of how we treated the data in detail, and analysis results. The other collects all the R code we wrote for each stage of data collection and analysis. We recommend the latter is used by those wishing to replicate our work and strongly encourage peers to use it and produce new meta-analyses in the social sciences.
\end{quote}

\vspace{.3cm}

\noindent \textbf{5)} Reviewer 2 writes: ``To facilitate interpretation of Figure 1, it would be more convenient to number each plot, and refer to it in the body text (pp. 10-11).''

\vspace{.3cm}

\noindent \textbf{Response:} 
\begin{quote}
    All plots in Figures 1 and 2 are now individually numbered, and referred to in the body text.
\end{quote}

We would like to thank Reviewer 2 for her/his helpful comments and suggestions.

\newpage
\bibliography{../article/references.bib}
\bibliographystyle{apalike}

\end{document}d
